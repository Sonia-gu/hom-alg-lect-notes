\documentclass[../main.tex]{subfiles}

\begin{document}
\section{Диаграмный поиск}
\begin{to_thr}[Фрейд-Митчелл]
Для любой малой абелевой категории существует полный, строгий и точный функтор в $R-mod$ для некоторого кольца $R$.
\end{to_thr}
Благодаря этой теореме многие утверждения для диаграмм можно доказывать рассматривая объекты абелевой категории как модули, а морфизмы --- как соответствующие гомоморфизмы. Приведём примеры.

\begin{to_suj}[Лемма о пяти]\label{B:five_lemma}
Пусть в абелевой категории дана коммутативная диаграмма с точными строками.
\begin{equation*}
    \begin{tikzcd}
	A & B & C & D & E \\
	{A'} & {B'} & {C'} & {D'} & {E'}
	\arrow[from=1-1, to=1-2]
	\arrow[from=1-2, to=1-3]
	\arrow[from=1-3, to=1-4]
	\arrow[from=1-4, to=1-5]
	\arrow[from=2-1, to=2-2]
	\arrow[from=2-2, to=2-3]
	\arrow[from=2-3, to=2-4]
	\arrow[from=2-4, to=2-5]
	\arrow["\alpha"', from=1-1, to=2-1]
	\arrow["\beta"', from=1-2, to=2-2]
	\arrow["\gamma"', from=1-3, to=2-3]
	\arrow["\delta"', from=1-4, to=2-4]
	\arrow["\varepsilon"', from=1-5, to=2-5]
\end{tikzcd}
\end{equation*}
Тогда если $\beta, \delta$ --- изоморфизмы, $\alpha$ --- эпиморфизм, $\varepsilon$ --- мономорфизм, то $\gamma$ --- изоморфизм.
\end{to_suj}
\begin{proof}
Рассмотрим правую часть диаграммы и покажем, что $\gamma$ --- эпиморфизм.
\begin{equation*}
    \begin{tikzcd}
	B & C & D & E \\
	{B'} & {C'} & {D'} & {E'} \\
	& c & d & 0 \\
	& {c'} & {d'} & 0
	\arrow["\beta", hook, two heads, from=1-1, to=2-1]
	\arrow["\gamma", from=1-2, to=2-2]
	\arrow["\delta", hook, two heads, from=1-3, to=2-3]
	\arrow["\varepsilon", hook, from=1-4, to=2-4]
	\arrow[from=1-1, to=1-2]
	\arrow[from=1-2, to=1-3]
	\arrow[from=1-3, to=1-4]
	\arrow[from=2-1, to=2-2]
	\arrow[from=2-2, to=2-3]
	\arrow[from=2-3, to=2-4]
	\arrow[maps to, from=4-2, to=4-3]
	\arrow[maps to, from=4-3, to=4-4]
	\arrow[from=3-4, to=4-4]
	\arrow[maps to, from=3-2, to=3-3]
	\arrow[maps to, from=3-3, to=3-4]
	\arrow[from=3-3, to=4-3]
	\arrow[from=3-2, to=4-2]
\end{tikzcd}
\end{equation*}{\color{red} Тут ошибочка}
Рассмотрим произвольный элемент $c'\in C'$. Из точности нижней строки его образом в $E'$ будет $0$. $\varepsilon$ --- мономорфизм, откуда его прообраз в $E$--- ноль. Из точности верхней последовательности у него будет прообраз $c\inC$. Это доказывает сюрьективность.

Теперь рассмотрим правую часть диаграммы. Докажем инъективность $\gamma$.
\begin{equation*}
    \begin{tikzcd}
	A & B & C & D \\
	{A'} & {B'} & {C'} & {D'} \\
	a & b & {x=0} & 0 \\
	{a'} & {b'} & 0 & 0
	\arrow["\alpha"', two heads, from=1-1, to=2-1]
	\arrow["\beta"', hook, two heads, from=1-2, to=2-2]
	\arrow["\gamma"', from=1-3, to=2-3]
	\arrow["\delta"', hook, two heads, from=1-4, to=2-4]
	\arrow[from=1-1, to=1-2]
	\arrow[from=1-2, to=1-3]
	\arrow[from=1-3, to=1-4]
	\arrow[from=2-1, to=2-2]
	\arrow[from=2-2, to=2-3]
	\arrow[from=2-3, to=2-4]
	\arrow[maps to, from=4-2, to=4-3]
	\arrow[maps to, from=4-1, to=4-2]
	\arrow[maps to, from=3-1, to=4-1]
	\arrow[maps to, from=3-1, to=3-2]
	\arrow[maps to, from=3-2, to=4-2]
	\arrow[from=3-2, to=3-3]
	\arrow[maps to, from=3-3, to=4-3]
	\arrow[maps to, from=4-3, to=4-4]
	\arrow[maps to, from=3-4, to=4-4]
	\arrow[from=3-3, to=3-4]
\end{tikzcd}
\end{equation*}
Пусть $x\in \Ker\gamma$. Образом нуля в $D'$ будет ноль. $\delta$ --- изооморфизм, откуда прообраз нуля в $D$ будет ноль. Из коммутативности правого квадрата и точности верхней последовательности $x$ имеет прообраз $b\in B$. Его образ $\beta(b) = b'$ из точности нижней последовательности и коммутативности тогда имеет прообраз $a'\in A'$. Т. к. $\alpha$ --- эпиморфизм, то $a'$ есть прообраз $a\in A$. Но верхняя строка точна, откуда $x=0$. Это доказывает инъективность $\gamma$.
\end{proof}

\begin{to_suj}[Лемма о змее]\label{B:snake_lemma}
Пусть в абелевой категории дана коммутативная диаграмма с точными строками.
\begin{equation*}
\begin{tikzcd}
	& A & B & C & 0 \\
	0 & {A'} & {B'} & {C'}
	\arrow[from=1-2, to=1-3]
	\arrow[from=1-3, to=1-4]
	\arrow[from=1-4, to=1-5]
	\arrow[from=2-1, to=2-2]
	\arrow[from=2-2, to=2-3]
	\arrow[from=2-3, to=2-4]
	\arrow["\alpha", from=1-2, to=2-2]
	\arrow["\beta", from=1-3, to=2-3]
	\arrow["\gamma", from=1-4, to=2-4]
\end{tikzcd}
\end{equation*}
Тогда существует длинная точная последовательность.
\begin{equation*}
    \begin{tikzcd}
	\Ker\alpha & \Ker\beta & \Ker\gamma & \coker\alpha & \coker\beta & \coker\gamma
	\arrow[from=1-1, to=1-2]
	\arrow[from=1-2, to=1-3]
	\arrow[from=1-3, to=1-4]
	\arrow[from=1-4, to=1-5]
	\arrow[from=1-5, to=1-6]
\end{tikzcd}
\end{equation*}
\end{to_suj}
\begin{proof}\textbf{Комедия в четырёх действиях}
    \begin{equation*}
        \begin{tikzcd}[sep=huge]
	& \Ker\alpha & \Ker\beta & \Ker\gamma \arrow["\delta", dashed, dlldd,
        "\delta",
        rounded corners,
        to path={ -- ([xshift=2ex]\tikztostart.east)
        |- (Z) [near end]\tikztonodes
        -| ([xshift=-2ex]\tikztotarget.west)
        -- (\tikztotarget)}]  \\
	& A & B & C & 0 \\
	0 & {A'} & {B'} & {C'} \\
	& \coker\alpha & \coker\beta & \coker\gamma
	\arrow["\alpha"'{pos=0.9}, from=2-2, to=3-2]
	\arrow["\beta"'{pos=0.9}, from=2-3, to=3-3]
	\arrow["\gamma"'{pos=0.9}, from=2-4, to=3-4]
	\arrow["{f_A}"{description}, from=2-2, to=2-3]
	\arrow["{f_B}"{description}, two heads, from=2-3, to=2-4]
	\arrow[from=2-4, to=2-5]
	\arrow[from=3-1, to=3-2]
	\arrow["{f_{A'}}"{description}, from=3-2, to=3-3]
	\arrow["{f_{B'}}"{description}, from=3-3, to=3-4]
	\arrow["{K_\alpha}"', hook, from=1-2, to=2-2]
	\arrow["{K_\beta}"', hook, from=1-3, to=2-3]
	\arrow["{K_\gamma}"', hook, from=1-4, to=2-4]
	\arrow[dashed, from=1-2, to=1-3]
	\arrow[dashed, from=1-3, to=1-4]
	\arrow[dashed, from=4-2, to=4-3]
	\arrow[dashed, from=4-3, to=4-4]
	\arrow["{C_\alpha}"', two heads, from=3-2, to=4-2]
	\arrow["{C_\beta}"', from=3-3, to=4-3]
	\arrow["{C_\gamma}"', from=3-4, to=4-4]
 \arrow[]
\end{tikzcd}
    \end{equation*}
    {\color{red} Добавить про нули в концах}
    \textbf{Построение морфизмов}
    
    Построим искомые морфизмы. Из коммутативности диаграммы $f_{A'}\circ \alpha\circ K_\alpha = \beta \circ f_A\circ K_\alpha = 0$, где равенство нулю следует из $\alpha\circ K_\alpha = 0$. Тогда по универсальному свойству ядра существует единственный морфизм $\Ker\alpha\to \Ker\beta$, делающий квадрат $\Ker\alpha$, $\Ker\beta$, $A$, $B$ коммутативным. Его мы и возьмём. Так строятся все морфизмы кроме "змеи" $\delta:\Ker\gamma\to \coker\alpha$.

    \textbf{Построение "змеи"}
    
    "Змею" мы построим диаграмным поиском.
    \begin{equation*}
       \begin{tikzcd}
	&&& \Ker\gamma &&&& {k_\gamma} \\
	&& B & C & 0 && b & {K_\gamma(k_\gamma)} \\
	0 & {A'} & {B'} & {C'} && {a'} & {\beta(b)=f_{A'}(a')} & {\gamma\circ K_\gamma(k_\gamma) = 0} \\
	& {\coker\alpha} &&&& {C_\alpha(a')}
	\arrow[two heads, from=2-3, to=2-4]
	\arrow[hook, from=1-4, to=2-4]
	\arrow[from=2-4, to=3-4]
	\arrow[from=2-3, to=3-3]
	\arrow[from=3-3, to=3-4]
	\arrow[hook, from=3-2, to=3-3]
	\arrow[from=3-1, to=3-2]
	\arrow[from=2-4, to=2-5]
	\arrow[from=3-2, to=4-2]
	\arrow[maps to, from=2-7, to=3-7]
	\arrow[maps to, from=2-8, to=3-8]
	\arrow[maps to, from=3-7, to=3-8]
	\arrow[maps to, from=2-7, to=2-8]
	\arrow[maps to, from=3-6, to=3-7]
	\arrow[maps to, from=3-6, to=4-6]
	\arrow[maps to, from=1-8, to=2-8]
\end{tikzcd}
    \end{equation*}
    Рассмотрим $k_\gamma\in\Ker\gamma$. По определению ядра $\gamma\circ K_\gamma(k_\gamma)=0$. Из сюрьективности $K_\gamma(k_\gamma)$ имеет прообраз $b\in B$. Тогда из коммутативности диаграммы и точности нижней строки имеем прообраз $f_{A'}(a') = \beta(b)$. Тогда искомый $\delta(k_\gamma) = C_\alpha(a')$.

    \textbf{Корректность определения "змеи"}
    
    Отметим, что определённый выше морфизм неоднозначен. Неоднозначность заключается в выборе $b$. Рассмотрим два различных $b$ и $b_1$. "Вычтем" соответствующие диаграммы друг из друга.
    \begin{equation*}
        \begin{tikzcd}
	&& 0 \\
	a & {b-b_1} & 0 \\
	{a' - a_1'} & {\beta(b-b_1)} & 0 \\
	{C_{\alpha}(a' - a_1') = 0}
	\arrow[from=1-3, to=2-3]
	\arrow[from=2-3, to=3-3]
	\arrow[from=2-2, to=2-3]
	\arrow[from=3-2, to=3-3]
	\arrow[from=3-1, to=3-2]
	\arrow[from=2-2, to=3-2]
	\arrow[from=2-1, to=2-2]
	\arrow[from=2-1, to=3-1]
	\arrow[from=3-1, to=4-1]
\end{tikzcd}
    \end{equation*}
    Из точности $b-b_1$ имеет прообраз $f_A(a) = b- b_1$, причем т. к. $f_{A'}$ --- мономорфизм имеем $\alpha(a) = a'-a_1'$. Тогда из коммутативности и по определению коядра соответственно получаем $C_\alpha(a'-a_1') = \alpha\circ C_\alpha(a'-a_1') = 0$.

    \textbf{Доказательство точности}

    Докажем точность в $\Ker\beta$ и $\Ker\gamma$. Для коядер доказательство аналогично.

    Пусть $k_\alpha \in \Ker\alpha$. Из коммутативности его образ в $\Ker\gamma$ будет $K_\gamma^{-1} \circ f_B\circ f_A\circ K_\alpha(k_\alpha)$, причём $f_B\circ f_A = 0$ из точности и $K_\gamma$ --- мономорфизм. Откуда образ в $\Ker\gamma$ будет ноль. Это доказывает вложение образа в ядро в члене $\Ker\beta$.
    \begin{equation*}
        \begin{tikzcd}
	{k_\alpha} & {k_\beta} & 0 \\
	{K_\alpha} & {f_A\circ K_\alpha(k_\alpha)} & {f_B\circ f_A\circ K_\alpha(k_\alpha)=0}
	\arrow[maps to, from=1-1, to=1-2]
	\arrow[maps to, from=2-1, to=2-2]
	\arrow[maps to, from=1-1, to=2-1]
	\arrow[maps to, from=1-2, to=1-3]
	\arrow[maps to, from=1-2, to=2-2]
	\arrow[maps to, from=2-2, to=2-3]
	\arrow[maps to, from=1-3, to=2-3]
\end{tikzcd}
    \end{equation*}

    Пусть $k_\beta$ лежит в ядре морфизма $\Ker\beta\to\Ker\gamma$. Тогда из точности верхней строки существует прообраз $f_A(a) = K_\beta(k_\beta)$. Из точности нижней строки и коммутативности диаграммы $\alpha(a) = 0$, откуда у $a$ по определению ядра существует прообраз $K_\alpha(k_\alpha) = a$. Т. к. $K_\beta$ --- мономорфизм получим, что $k_\alpha$ --- прообраз $k_\beta$. Это доказывает вложение ядра в образ и, следовательно, точность в члене $\Ker\beta$.
    \begin{equation*}
        \begin{tikzcd}
	{k_\alpha} & {k_\beta} & 0 \\
	a & {K_\beta(k_\beta)} & 0 \\
	0 & 0
	\arrow[maps to, from=1-2, to=1-3]
	\arrow[maps to, from=1-2, to=2-2]
	\arrow[maps to, from=2-2, to=3-2]
	\arrow[maps to, from=1-3, to=2-3]
	\arrow[maps to, from=2-2, to=2-3]
	\arrow[maps to, from=2-1, to=2-2]
	\arrow[maps to, from=3-1, to=3-2]
	\arrow[maps to, from=2-1, to=3-1]
	\arrow[maps to, from=1-1, to=2-1]
	\arrow[maps to, from=1-1, to=1-2]
\end{tikzcd}
    \end{equation*}

    Пусть теперь $k_\gamma \in \Ker\delta$. Вспомним построение $\delta$, где теперь $C_\alpha(a')=0$. Тогда по определению коядра существует прообраз $\alpha(a) = a'$, причём $f_B\circ f_A (a) = 0$ и $\beta\circ f_A(a) = \beta(b)$. Это позволяет "вычесть" из правой части диаграммы образы $a$.
    \begin{equation*}
        \begin{tikzcd}
	{f_A(a)} && {k_\gamma} & {k_\beta} & {k_\gamma} \\
	a & b & {K_\gamma(k_\gamma)} & {b-f_A(a)} & {K_\gamma(k_\gamma)} \\
	{a'} & {\beta(b) = \beta\circ f_A(a)} & 0 & 0 & 0 \\
	{C_\alpha(a')}
	\arrow[maps to, from=1-3, to=2-3]
	\arrow[maps to, from=2-1, to=3-1]
	\arrow["{f_A}", maps to, from=2-1, to=1-1]
	\arrow[maps to, from=3-1, to=3-2]
	\arrow[maps to, from=2-2, to=3-2]
	\arrow[maps to, from=2-3, to=3-3]
	\arrow[maps to, from=2-2, to=2-3]
	\arrow[maps to, from=3-2, to=3-3]
	\arrow[maps to, from=3-1, to=4-1]
	\arrow[maps to, from=1-4, to=2-4]
	\arrow[maps to, from=1-4, to=1-5]
	\arrow[maps to, from=2-4, to=3-4]
	\arrow[maps to, from=1-5, to=2-5]
	\arrow[maps to, from=2-5, to=3-5]
	\arrow[maps to, from=2-4, to=2-5]
	\arrow[maps to, from=3-4, to=3-5]
\end{tikzcd}
    \end{equation*}
    Тогда по определению ядра существует прообраз $K_\beta (b-f_A(a))$. Из коммутативности и инъективности $K_\gamma$ получим, что $k_\beta$ --- прообраз $k_\gamma$. Это доказывает вложение ядра в образ в члене $\Ker\gamma$.

    Теперь пусть $k_\beta\in \Ker\beta$. По определению ядра получим нули в $B'$ и $C'$. Из инъективности $f_{A'}$ им соответствует 0 в $A'$. Соответственно построение морфизма даёт $C_\alpha (0) = 0$. Это доказывает вложение образа в ядро и, следовательно, точность в члене $\Ker\gamma$.
    \begin{equation*}
        \begin{tikzcd}
	& {k_\beta} & {k_\gamma} \\
	& {K_\beta(k_\beta)} & {K_\gamma(k_\gamma)} \\
	0 & 0 & 0 \\
	0
	\arrow[maps to, from=1-2, to=2-2]
	\arrow[from=1-2, to=1-3]
	\arrow[maps to, from=1-3, to=2-3]
	\arrow[maps to, from=2-3, to=3-3]
	\arrow[maps to, from=2-2, to=2-3]
	\arrow[maps to, from=2-2, to=3-2]
	\arrow[maps to, from=3-2, to=3-3]
	\arrow[maps to, from=3-1, to=3-2]
	\arrow[maps to, from=3-1, to=4-1]
\end{tikzcd}
    \end{equation*}
\end{proof}

\begin{to_suj}[Лемма о зигзаге]\label{B:zigzag_lemma}
Пусть есть короткая точная последовательность.
\begin{equation*}
    \begin{tikzcd}
	0 & {K^\bullet} & {L^\bullet} & {M^\bullet} & 0
	\arrow[from=1-1, to=1-2]
	\arrow["f", from=1-2, to=1-3]
	\arrow["g", from=1-3, to=1-4]
	\arrow[from=1-4, to=1-5]
\end{tikzcd}
\end{equation*}
Тогда существуют связующие морфизмы $\delta^i: H^i(M^\bullet) \to H^{i+1}(K^\bullet)$, делающие следующую последовательность точной.
\begin{equation*}
    \begin{tikzcd}
	&& \ldots \\
	{H^{i-1}(K^\bullet)} & {H^{i-1}(L^\bullet)} & {H^{i-1}(M^\bullet)} \\
	{H^i(K^\bullet)} & {H^i(L^\bullet)} & {H^i(M^\bullet)} \\
	{H^{i+1}(K^\bullet)} & {H^{i+1}(L^\bullet)} & {H^{i+1}(M^\bullet)} \\
	\ldots
	\arrow[from=1-3, to=2-1]
	\arrow["{f^*}"', from=2-1, to=2-2]
	\arrow["{g^*}", from=2-2, to=2-3]
	\arrow["{\delta^{i-1}}"{description}, from=2-3, to=3-1]
	\arrow["{f^*}"', from=3-1, to=3-2]
	\arrow["{g^*}", from=3-2, to=3-3]
	\arrow["{\delta^i}"{description}, from=3-3, to=4-1]
	\arrow["{f^*}"', from=4-1, to=4-2]
	\arrow["{g^*}", from=4-2, to=4-3]
	\arrow[from=4-3, to=5-1]
\end{tikzcd}
\end{equation*}
\end{to_suj}
\end{document}