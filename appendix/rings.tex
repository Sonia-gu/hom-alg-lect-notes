\documentclass[../main.tex]{subfiles}
\begin{document}
\section{Что-то о нётеровых кольцах}
Несколько примеров фильтрации из коммутативной алгебры.
\begin{to_ex}[Фильтрация кольца]
Убывающей мкльтипликативной фильрацией кольца $R$ называется убывающая последовательность идеалов вида 
\[
R = I_0 \supset I_1 \supset I_2 \supset \ldots
\]
Удовлетворяющая условию $I_i I_j \subset I_{i+j}$ $\forall$ $i, j$.
Эта конструкция чаще всего используется, в случае, когда $I_j = I^j$ -- степени одного идеала $I$, это называется $I$-адической фильтрацией. В приложениях чаще всего встречается ситуация локального нётерова кольца 
 и его максимального идеала.\\
 Полезно обобщить эту конструкцию на $R-mod$  иизучать фильтрации модулей
 \[M \supset IM \supset I^2 M \supset \ldots\]
 Однако, пересекая члены такой фильтрации с некоторым подмодулем $M' \subset M$ в общем случае мы не получим $I$-адическую фильтрацию $M'$.
\end{to_ex}
\end{document}