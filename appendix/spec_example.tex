\documentclass[../main.tex]{subfiles}

\begin{document}
\section{Спектральная последовательность. Дополнительные примеры. }
\begin{to_ex}[Элементарный пример, нефильтрованный]
Самым простым примером спектральной последовательности является любой комплекс $K_\bullet$ $\in$ $Kom(\A)$. Он естественно имеет диффернциал $d_i^{K}\colon K_i \to K_{i+1}$, поэтому на нулевом листе $d_0 = d^K$. Пусть $E^\bullet_0 = K_\bullet$, тогда $E^\bullet_1 = H^\bullet(C_\bullet)$, дифференциал индуцирует нулевые отображения на когомологиях, поэтому на первом листе мы имеем $d_1 = 0$. Из этого следует, что $E_2 = E^\bullet_{\infty}$ и $d_n = 0$ $\forall$ $n\ge 2$, таким образом, получаем следующую спектральную последовательность:
\begin{itemize}
    \item $E_0 = K_\bullet$
    \item  $E_r = H(K_\bullet)$ $\forall$ $r\ge1$
\end{itemize}
Такая спектральная последовательность стабилизируется на первом листе, так как нетривиальные дифференциалы присутствовали лишь на нулевом. 
\end{to_ex}
\subsection{Доказательство пять-леммы с помощью спектральной последовательности}
\begin{to_ex}[Пять-лемма]
Спектральная последовательность по столбцам сходится к $0$, поэтому средняя стрелка на второй диаграмме ниже -- изоморфизм.
\bee
\begin{tikzcd}
    0\arrow[r]&A' \arrow[r]& B' \arrow[r]& C' \arrow[r]& 0\\
    0\arrow[r]&A \arrow[r]\arrow[u, "\cong"]& B \arrow[r]\arrow[u, "\beta"] &C \arrow[r]\arrow[u, "\cong"]& 0
\end{tikzcd}
\eee
\bee
\begin{tikzcd}
    0\arrow[r]&0 \arrow[r]& \coker\beta \arrow[r]& 0 \arrow[r]& 0\\
    0\arrow[r]&0 \arrow[r]\arrow[u, "\cong"]& \ker\beta \arrow[r]\arrow[u] &0 \arrow[r]\arrow[u, "\cong"]& 0
\end{tikzcd}
\eee
\end{to_ex}
\subsection{Доказательство леммы о змее с помощью спектральной последовательности}
Докажем лемму о змее \ref{B:snake_lemma}, используя спектральную последовательность. Диаграмма в этой лемме представляет собой двойной комплекс $C^{\bullet\bullet}$ с точными строками в абелевой категории. 
\bee
\begin
\eee
\begin{equation*}
\begin{tikzcd}
	0 & C^{00} & C^{10} & C^{20} & 0 \\
	0 & C^{01} & C^{11} & C^{21} & 0
	\arrow[from=1-1, to=1-2]
	\arrow[from=1-2, to=1-3]
	\arrow[from=1-3, to=1-4]
	\arrow[from=1-4, to=1-5]
	\arrow[from=2-1, to=2-2]
	\arrow[from=2-2, to=2-3]
	\arrow[from=2-3, to=2-4]
	\arrow[from=2-4, to=2-5]
	\arrow["\alpha", from=1-2, to=2-2]
	\arrow["\beta", from=1-3, to=2-3]
	\arrow["\gamma", from=1-4, to=2-4]
\end{tikzcd}
\end{equation*}
Для такого комплекса существуют две спектральные последовательности, которые будут сходиться к когомологиям тотального комплекса $H^{p+q}(Tot(C^{\bullet\bullet}))$.\\
Нулевые листы этих последовательностей, естественно, $E_0^{pq} = C^{pq}$ и $E_0^{pq} = C^{qp}$:
\begin{table}[h]
\begin{minipage}{.4\linewidth}
\centering
\begin{tabular}{ |c c c c c c } 
 0 & $\rightarrow$ & 0 & $\rightarrow$ & 0 \\ 
 $C^{20}$ & $\rightarrow$ & $C^{21}$ & $\rightarrow$ & 0 \\
 $C^{10}$ & $\rightarrow$ & $C^{11}$ & $\rightarrow$ & 0 \\
 $C^{00}$ & $\rightarrow$ & $C^{01}$ & $\rightarrow$ & 0 \\
 0 & $\rightarrow$ & 0 & $\rightarrow$ & 0 \\ 
 \hline
\end{tabular}
\end{minipage}
\begin{minipage}{.4\linewidth}
\centering
\begin{tabular}{ |c c c c c c c c} 
 0 $\rightarrow$& 0 & $\rightarrow$ & 0 & $\rightarrow$ & 0 & $\rightarrow$ & 0\\ 
 0 $\rightarrow$& $C^{01}$ & $\rightarrow$ & $C^{11}$ & $\rightarrow$ & $C^{21}$ & $\rightarrow$ & 0 \\
 0 $\rightarrow$& $C^{00}$ & $\rightarrow$ & $C^{10}$ & $\rightarrow$ & $C^{20}$ & $\rightarrow$ & 0\\
 \hline
\end{tabular}
\end{minipage}
\end{table}

\begin{table}[h]
\begin{minipage}{.4\linewidth}
\centering
\begin{tabular}{ |c c c } 
 0 & 0 & 0 \\ 
 $\uparrow$   &	 $\uparrow$   & $\uparrow$   \\
 $\ker\gamma$ & $\coker\gamma$ & 0 \\
 $\uparrow$   &	 $\uparrow$   & $\uparrow$   \\
 $\ker\beta$ & $\coker\beta$ & 0 \\
 $\uparrow$   &	 $\uparrow$   & $\uparrow$   \\
 $\ker\alpha$ & $\coker\alpha$ & 0 \\
 $\uparrow$   &	 $\uparrow$   & $\uparrow$   \\
 0 & 0 & 0 \\ 
 \hline
\end{tabular}
\end{minipage}
\begin{minipage}{.4\linewidth}
\centering
\begin{tabular}{ |c c c c c} 
 0 & 0 & 0 & 0 & 0\\ 
 0 & 0 & 0 & 0 & 0\\
 0 & 0 & 0 & 0 & 0 \\
 \hline
\end{tabular}
\end{minipage}
\end{table}

\begin{table}[h]
\begin{minipage}{.3\linewidth}
\centering
\begin{tabular}{ |c  } 
\bee
\begin{tikzcd}
 0 & 0  \\ 
 \ker\gamma/\ker\beta & 0 \\
 0 & 0\arrow[luu] \\
 0 & \ker(\coker\alpha \to \coker\beta)\arrow[luu, "\delta^{-1}"']
\end{tikzcd}
\eee\\
 \hline
\end{tabular}
\end{minipage}
\end{table}

\subsection{Доказательство расширенной пять-леммы с помощью спектральной последовательности}
\bee
\begin{tikzcd}
    A' \arrow[r]& B' \arrow[r]& C' \arrow[r]&D' \arrow[r]&E' \\
    A \arrow[r]\arrow[u, two heads, "\alpha"]& B \arrow[r]\arrow[u, "\cong"] &C \arrow[r]\arrow[u, "f"]&D \arrow[r]\arrow[u, "\cong"]&E \arrow[u, hook, "\varepsilon"]
\end{tikzcd}
\eee
$E_1^{\bullet \bullet}$ с фильтрацией по строкам и по столбцам:
\begin{table}[h]
\begin{minipage}{.4\linewidth}
\centering
\begin{tabular}{ |c  } 
\bee
\begin{tikzcd}
    A' & 0 & 0 &0 &E' \\
    A \arrow[u]& 0 \arrow[u] &0 \arrow[u]&0 \arrow[u]&E \arrow[u]
\end{tikzcd}
\eee\\
 \hline
\end{tabular}
\end{minipage}
\begin{minipage}{.4\linewidth}
\centering
\begin{tabular}{ |c c c c c} 
\bee
\begin{tikzcd}
0 \arrow[r]                 & 0 \arrow[r] & ? \arrow[r] & 0 \arrow[r] & \ker\alpha \\
\coker\varepsilon \arrow[r] & 0 \arrow[r] & ? \arrow[r] & 0 \arrow[r] & 0         
\end{tikzcd}
\eee\\
 \hline
\end{tabular}
\end{minipage}
\end{table}
Так как обе спектральные последовательности сходятся к когомологиям тотального комплекса $?=0$, т. е. $f$ -- изоморфизм.
\begin{to_ex}[Эйлерова характеристика]
$\chi \colon \Ab \to G$, $G$ -- абелева группа. 
Фильтрация длины 1
\bee
\begin{tikzcd}[sep=small]
0 \arrow[r]& A \arrow[r] & B \arrow[r] &C \arrow[r] & 0
\end{tikzcd}
\qquad \chi(B) = \chi(A) + \chi(C)
\eee
\[
\chi (K^\bullet) = \sum_{}^{}(-1)^n \chi (K^\bullet) = \sum (-1)^n \chi (H^n(K^\bullet))
\]
$\chi (E^n) = ?$\\
$E^{pq}_r \Rightarrow E^n$
Вычислим альтернированную сумму на листе $r$
\[E_{pq}^{\bullet} = \underset{p+q = \bullet}{\oplus} E_r^{pq}\]
\[\chi (E^{\bullet \bullet}_r) = \sum (-1)^k \chi(H^k(E_r^{\bullet})) = \chi(E_{r+1}^{\bullet}) =\ldots = \chi(E_{\infty}^{\bullet})\]
\[\chi(E^n) = \sum \chi (F^pE^n / F^{p+1}E^n) = \chi(E^{\bullet \bullet}_{\infty})\]
\end{to_ex}
\end{document}

