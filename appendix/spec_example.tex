\documentclass[../main.tex]{subfiles}

\begin{document}
\section{Спектральная последовательность. Дополнительные примеры. }
\begin{to_ex}[Элементарный пример, нефильтрованный]
Самым простым примером спектральной последовательности является любой комплекс $K_\bullet$ $\in$ $Kom(\A)$. Он естественно имеет диффернциал $d_i^{K}\colon K_i \to K_{i+1}$, поэтому на нулевом листе $d_0 = d^K$. Пусть $E^\bullet_0 = K_\bullet$, тогда $E^\bullet_1 = H^\bullet(C_\bullet)$, дифференциал индуцирует нулевые отображения на когомологиях, поэтому на первом листе мы имеем $d_1 = 0$. Из этого следует, что $E_2 = E^\bullet_{\infty}$ и $d_n = 0$ $\forall$ $n\ge 2$, таким образом, получаем следующую спектральную последовательность:
\begin{itemize}
    \item $E_0 = K_\bullet$
    \item  $E_r = H(K_\bullet)$ $\forall$ $r\ge1$
\end{itemize}
Такая спектральная последовательность стабилизируется на первом листе, так как нетривиальные дифференциалы присутствовали лишь на нулевом. 
\end{to_ex}

\end{document}

