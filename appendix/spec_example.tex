\documentclass[../main.tex]{subfiles}

\begin{document}
\section{Спектральная последовательность. Дополнительные примеры. }
\begin{to_ex}[Элементарный пример, нефильтрованный]
Самым простым примером спектральной последовательности является любой комплекс $K_\bullet$ $\in$ $Kom(\A)$. Он естественно имеет диффернциал $d_i^{K}\colon K_i \to K_{i+1}$, поэтому на нулевом листе $d_0 = d^K$. Пусть $E^\bullet_0 = K_\bullet$, тогда $E^\bullet_1 = H^\bullet(C_\bullet)$, дифференциал индуцирует нулевые отображения на когомологиях, поэтому на первом листе мы имеем $d_1 = 0$. Из этого следует, что $E_2 = E^\bullet_{\infty}$ и $d_n = 0$ $\forall$ $n\ge 2$, таким образом, получаем следующую спектральную последовательность:
\begin{itemize}
    \item $E_0 = K_\bullet$
    \item  $E_r = H(K_\bullet)$ $\forall$ $r\ge1$
\end{itemize}
Такая спектральная последовательность стабилизируется на первом листе, так как нетривиальные дифференциалы присутствовали лишь на нулевом. 
\end{to_ex}
\subsection{Доказательство леммы о змее с помощью спектральной последовательности}
Докажем лемму о змее \ref{B:snake_lemma}, используя спектральную последовательность. Диаграмма в этой лемме представляет собой двойной комплекс $C^{\bullet\bullet}$ с точными строками в абелевой категории. 
\bee
\begin
\eee
\begin{equation*}
\begin{tikzcd}
	0 & C^{00} & C^{10} & C^{20} & 0 \\
	0 & C^{01} & C^{11} & C^{21} & 0
	\arrow[from=1-1, to=1-2]
	\arrow[from=1-2, to=1-3]
	\arrow[from=1-3, to=1-4]
	\arrow[from=1-4, to=1-5]
	\arrow[from=2-1, to=2-2]
	\arrow[from=2-2, to=2-3]
	\arrow[from=2-3, to=2-4]
	\arrow[from=2-4, to=2-5]
	\arrow["\alpha", from=1-2, to=2-2]
	\arrow["\beta", from=1-3, to=2-3]
	\arrow["\gamma", from=1-4, to=2-4]
\end{tikzcd}
\end{equation*}
Для такого комплекса существуют две спектральные последовательности, которые будут сходиться к когомологиям тотального комплекса $H^{p+q}(Tot(C^{\bullet\bullet}))$. Нулевые листы этих последовательностей, естественно, $E_0^{pq} = C^{pq}$ и $E_0^{pq} = C^{qp}$
\subsection{Что-то о нётеровых кольцах}
Несколько примеров фильтрации из коммутативной алгебры.
\begin{to_ex}[Фильтрация кольца]
Убывающей мкльтипликативной фильрацией кольца $R$ называется убывающая последовательность идеалов вида 
\[
R = I_0 \supset I_1 \supset I_2 \supset \ldots
\]
Удовлетворяющая условию $I_i I_j \subset I_{i+j}$ $\forall$ $i, j$.
Эта конструкция чаще всего используется, в случае, когда $I_j = I^j$ -- степени одного идеала $I$, это называется $I$-адической фильтрацией. В приложениях чаще всего встречается ситуация локального нётерова кольца 
 и его максимального идеала.\\
 Полезно обобщить эту конструкцию на $R-mod$  иизучать фильтрации модулей
 \[M \supset IM \supset I^2 M \supset \ldots\]
 Однако, пересекая члены такой фильтрации с некоторым подмодулем $M' \subset M$ в общем случае мы не получим $I$-адическую фильтрацию $M'$.
\end{to_ex}
\end{document}

