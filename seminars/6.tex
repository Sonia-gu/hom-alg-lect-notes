\documentclass[../main.tex]{subfiles}
\begin{document}
\section{Строгое и полное вложение в производную категорию}
Пусть $\A$ -- абелева категория. Напомним результаты, полученные при её локализации по разным классам морфизмов:
\begin{align*}
    \mathbf{Ob}(\D{A}) = \mathbf{Ob}(\K{A}) = \mathbf{Ob}( Kom(\A))
\end{align*}
\bee
\begin{tikzcd}[sep=small]
    \A \arrow[r, hook]& Kom^{*}(\A) \arrow[r, two heads]& \Kstar{A} \arrow[r, two heads, "Q_{\A}"]& \Dstar{A}
\end{tikzcd}
\eee
\[
* \in \lbrace \varnothing, +, -, \flat\rbrace
\]
При локализации категории, несмотря на формальное сохранение объектов, \textit{меняется структура категории}, так как внутри категории мы работаем с классами эквивалентоности объектов, а при локализации меняется отношение "быть изоморфным". 
\begin{to_com}[почему проективный объект проективный]
Если мы имеем эпиморфизм между некоторым объектом $A$ и проективным объктом $P$, то $P$ отщипляеся как прямое слагаемое в $A$. В линейной алгебре это соответствует проекции на некоторое подпространство и вложению его как прямого слагаемого.
    \bee
        \begin{tikzcd}
             & P \arrow[d, "id"]\arrow[ld, dashed, "s"']\\
        A\arrow[r, two heads, "\pi"]& P
        \end{tikzcd}
    \eee
    \[
    s\pi \colon A \mapsto A
    \]
    \[
    A \cong A' \oplus P
    \]
\end{to_com}
\subsection{Каноническое вложение в производную категорию}
Существует каноническое вложение в производную категорию $\mathbf{\imath} \colon \A \hookrightarrow \Dstar{A}$, которое сопоставляет каждому объекту исходной категории комплекс с нетривиальным когомологиями в нулевом члене и все изоморфные ему, в частности, все точные резольвенты. 
\begin{to_claim}
\label{full_i}
$\mathbf{\imath} \colon \A \hookrightarrow \Dstar{A}$ -- строгий и полный.\footnote{строгая полнота вложений $\mathbf{\imath}\colon \A \to Kom(A)$ и $\mathbf{\imath}\colon \A \to \Kstar{A}$ очевидна, так как любой морфизм между объектами задаёт морфизм соответствующих комплексов, а все гомотопии таких комплексов пропускаются через ноль, поэтому нет нетривиальных гомотопий}
\[
\Hom_{\A}(A, B) \cong \Hom_{\D{A}}(A[0], B[0])
\]
\end{to_claim}
\begin{proof}
$\blacktriangle$\textbf{mono}(inj)
Возьмём $f \in \Hom_{\A}(A, B)$ такой, что $Q_{\A}(f) = 0$. Ему в производной категории будет соответствовать домик, эквивалентный нулевому домику. 
\bee
\begin{tikzcd}[sep=small]
    & A[0] \arrow[ld, "id"']\arrow[rd, "f"]& \\
    A[0]& & B[0]
\end{tikzcd} 
\qquad
\begin{tikzcd}[sep=small]
    & K^\bullet \arrow[ld, "g"']\arrow[rd, "fg"]& \\
    A[0]& & B[0]
\end{tikzcd} 
\eee
Применим функтор когомологий 
\[
0\quad \overset{fg\sim0}{=}\quad H^0 (fg) = H^0(f) H^0(g)\quad \overset{g-qis}{=}\quad H^0(f) 
\]
Для провекри сюръективности нам понадобится некоторое дополнительное знание о производной категории, а именно, определения следующих фунткоров, действующих на производной категории:
\begin{to_def}
Функтор сдвига
\begin{align*}
\mathcal{T}\colon \Dstar{A}& \to \Dstar{A}\\
 K^\bullet& \mapsto K^\bullet[1]
\end{align*}
\end{to_def}
\begin{to_def}
Функторы обрезания $\tau_{  \le\ge i}\colon \D{A} \to \D{A}$
\[
H^k(\taule{i} K^\bullet) = \begin{cases}H^k(K^\bullet) &, k \le i \\ 0 &, k>i \end{cases}
\]
\bee
\begin{tikzcd}[sep=small]
    \taule{i}K^\bullet &\ldots \arrow[r]&K^{i-1} \arrow[r]\arrow[d] & Z^{i}(K^\bullet) \arrow[r]\arrow[d]& 0\arrow[d]\arrow[r]&\ldots \\
    K^\bullet &\ldots \arrow[r]&K^{i-1} \arrow[r] & K^{i} \arrow[r]& K^{i+1}\arrow[r]&\ldots
\end{tikzcd}
\qquad
\begin{tikzcd}[sep=small]
    K^\bullet &\ldots\arrow[r]& K^{i-2} \arrow[r]\arrow[d] & K^{i-1} \arrow[r]\arrow[d]& K^{i+1}\arrow[d]\arrow[r]&\ldots \\
    \tauge{i}K^\bullet &\ldots\arrow[r]& 0 \arrow[r] & B^{i}(K^\bullet) \arrow[r]& K^{i+1}\arrow[r]&\ldots
\end{tikzcd}
\eee
\end{to_def}
Теперь можем доказывать сюръективность.
$\blacktriangle$\textbf{epi}(surj) Покажем, что, заданному в производной категории домику, будет соответствовать морфизм объектов в исходной категории. Так как $K^\bullet$ $A[0]$ квазиизоморфны, то у $K^\bullet$ есть только одна нетривиальная когмология. Применяя к $K^\bullet$ последовательно функтор обрезания, мы получим ещё один квазиизоморфный комплекс. Так как мы в производной категории, то стрелки соответствующие квазиизоморфизмам между этими комплексами можно обращать, а в силу квазиизоморфности $A[0]$ и $K^\bullet$ мы получаем эквивалентный домик, который соответствует морфизму в исходной категории.
\begin{align*}
    K^\bullet \rightarrow \tauge{0}K^\bullet \leftarrow \taule{0}\tauge{0}K^\bullet\\
    K^\bullet \leftrightarrows \taule{0}K^\bullet \leftrightarrows \taule{0}\tauge{0}K^\bullet
\end{align*}
\begin{align*}
\taule{0}\tauge{0}K^\bullet  \overset{qis}{\sim}K^\bullet \overset{qis}{\sim} A[0]
\end{align*}
\bee
\begin{tikzcd}[sep=small]
    & K^\bullet \arrow[ld, "qis"']\arrow[rd]& \\
    A[0]& & B[0]
\end{tikzcd} 
\quad \sim \quad
\begin{tikzcd}[sep=small]
    & A[0] \arrow[ld, "id"']\arrow[rd, "f"]& \\
    A[0]& & B[0]
\end{tikzcd}
\eee
\end{proof}
\subsection{Мотивировка}
Имея каноническое вложение, мы хотим по функтору мужду исходными аддитивными категориями определить функтор между производными категориями. При вложении в категорию комплексов и гомотопическую категорию комплексов, функтор корректно
продожается почленным действием на комплексы, так как, помимо всего прочего, он явояется гомоморфизмом абелевых групп морфизмов между объектами. То есть, если в $\A$ гомоморфизм распадался в сумму композиций, то и в $\B$ он будет распадаться в сумму композиций. Переход к производным категориям оказывается более сложным, потому что произвольный функтор не обязан сохранять отношение квазиизоморфности (не все функторы точные). Существует аналогия между гомологической алгеброй и линейной. Точные тройки играют в этой аналогии роль суммы элементов (группа Гротендика категории $K_0(\A)$), в которой также может быть введён базис), а функтор $\Hom$ (эйлерова характеристика $\chi$) -- роль скалярного произведения. Следуя этой логике, можно сказать, что, если точные тройки(выделенные треугольники) играют роль сумм, то точный функтор играет роль линейного отображения.\\
Будем требовать от функтора $\F$ точности хотя бы с одной стороны. Пусть, например, он точен справа. Этот неточный функтор мы будем "приближать" точными, отсюда и возникает понятие "производный" функтор.
\bee
    \begin{tikzcd}[sep=small]
        0 \arrow[r]& \F{}(A) \arrow[r]& \F{}(B) \arrow[r]& \F{}(C)\arrow[r]& R^1\F{}(A)\arrow[r]& R^1\F{}(B)\arrow[r]&R^1\F{}(C)\arrow[r]& \ldots
    \end{tikzcd}
\eee
Сформулируем следующие утверждения для точного функтора 
\begin{to_claim}
 Точный функтор сохраняет отношение квазиизоморфности. 
\end{to_claim}
\begin{to_claim}
 Почленное применение $\DF{F}$ корректно определено. 
\end{to_claim}
\begin{to_com}
    Для морфизма комплексов $f\colon K^\bullet\to L^\bullet$ определена следующая бесконечная последовательность морфизмов, каждая композиция в которой гомотопна 0 : 
    \bee \begin{tikzcd}[sep=small] K^\bullet \arrow[r]& L^\bullet \arrow[r]& C(f) \arrow[r]& K^\bullet[1]\end{tikzcd}\eee
\end{to_com}
\begin{to_claim}
 $\DF{F}$ точен как функтор между производными категориями, то есть выделенные трегуольники он переводит в выделенные.
\end{to_claim}
\begin{to_claim}
    Пусть $\F$ -- точен, а $K^\bullet$ -- ацикличен, тогда $\F{}(K^\bullet)$ -- ацикличен.
\end{to_claim}
\begin{proof}
    Ацикличный комплекс состоит из композиций точных троек вида 
    \bee
        \begin{tikzcd}
            0\arrow[r]& B^i(K^\bullet)\arrow[r, hook, "e^i"]&Z^i(K^\bullet)\arrow[r, two heads, "p^i"]&B^{i+1}(K^\bullet)\arrow[r]&0& d^i = p^i e^i \\
            0\arrow[r]& B^i(\F{K^\bullet})\arrow[r, hook, "\F e^i"]&Z^i(\F{K^\bullet})\arrow[r, two heads, "\F p^i"]& B^{i+1}(\F{K^\bullet})\arrow[r]&0& \F d^i = \F p^i \F e^i \\
        \end{tikzcd}
    \eee
    Конус квазиизоморфизма ацикличен. Поскольку функтор $\F$ -- сохраняет прямые суммы, и мы можем применять его почленно, а дифференциал в конусе определялся изоморфизмом, то существует канонический изоморфизм $\F(C(f))\cong C(\F(f))$. Оба конуса ацикличны, оба морфизма -- изоморфизмы.
\end{proof}
\subsection{Классический производный функтор, класс преспособленных объектов}
\begin{to_def}
    $\mathcal{R} \subset \mathbf{Ob}(\A)$ -- класс объектов, приспособленный к отчному слева функтору $\F$, если 
    \begin{itemize}
        \item $\F$ переводит ацикличные ограниченные комплексы из $\Kl{R}$ в ацикличные
        \item класс $\mathcal{R}$ -- достаточно большой, то есть $\forall A \in \A$ $\exists R \in \mathcal{R}$ $\colon$ $A \hookrightarrow R$ \footnote{Если бы $\F$ был точен справа, мы бы потребовали, чтобы любой объект категории являлся факторобъектом объекта из $\mathcal{R}$}
    \end{itemize}
\end{to_def}
$\mathbf{\NB}$ Приспособленный класс объектов определён неоднозначно. Например, если функтор точен, то любой достаточно большой класс объектов категории будет приспособленным. Этот факт существенно усложняет построение производного функтора.
\begin{to_def}
$\mathcal{R}$-резольвентой будем называть квазиизоморфный комплекс с членами из $\mathcal{R}$.\footnote{Существует эквивалентный подход к определению производного функтора, не использующий приспособленный класс, а только проективные и инъективные объекты.}
\end{to_def}
\begin{to_claim}
Существует эквивалентность категорий
\[\phi \colon \Dr{A} \cong \Kr{R}[QIS_{\mathcal{R}}^{-1}]\]
\end{to_claim}
\begin{to_claim}
 $QIS_{\mathcal{R}}$ являются локализующим семейством в $\Kl{R}$
\end{to_claim}
\begin{proof}
Конус квазиизоморфизма между комплексами приспособленных объектов будет приспособлен, так как при взятии суммы будет сохраняться ацикличность, а функтор аддитивный 
    \[QIS_{\mathcal{R}} \ni f\colon K^\bullet \to L^\bullet \qquad K^\bullet, L^\bullet \in \Kl{R} \quad \Rightarrow \quad C(f) \in \Kl{R}\]
\end{proof}
\end{document}