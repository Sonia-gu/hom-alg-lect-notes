\documentclass[../main.tex]{subfiles}
\begin{document}
В предыдущих сериях мы рассматривали двойной ограниченный комплекс $L^{ij}$, его тотальный комплекс $Tot^{\oplus}(K^{\bullet\bullet})^n = \myunder[]{\bigoplus}{p+q=n}K^{pq}$ и соответствующие фильтрации:
\[
^I F^p(Tot L)^n = \myunder[]{\bigoplus}{i+j = n,\\ i\ge p}L^{ij}
\]
\[
^{II} F^q(Tot L )^n = \myunder[]{\bigoplus}{i+j = n,\\ i\ge q}L^{ij}
\]
\begin{align*}
    \exists \text{с. п.} ^I E_r^{pq} = H^p_I(H_{II}^q) \Rightarrow H^{p+q}(Tot L)
\end{align*}
Если мы делаем фильтрацию типа $I$, дифференциал бьёт в бок, на втором листе будут стоять горизонтальные когомологии.
\begin{align*}
E_r^{pq} \colon \qquad Z_1^{pq} &= (d_I + d_II)^{-1}(F^{p+1}Tot ^{p+q+1}L)\bigcap F^p Tot^{p+q}L\\ \\
&=(d_I + d_{II})^{-1}\left(\myunder[]{\bigoplus}{$i+j = p+q$\\$ i\ge p+1$}L^{ij}\right)\bigcap \left(\myunder[]{\bigoplus}{$i+j = p+q$\\$i\ge p$}L^{ij}\right)
= \ker d_{II}^{pq}\bigoplus\left(\myunder[]{\bigoplus}{$i+j=p+q$\\$i\ge p+1$}L^{ij}\right)
\end{align*}
\begin{align*}
E_r^{pq} = \dfrac{Z_2^{pq}}{Z_{r-1}^{p+1, q-1} + dZ_{r-1}^{p-r+1;q+r-2}} \overset{r=1}{=}
\dfrac{\ker d^{pq}\bigoplus\left(\myunder[]{\bigoplus}{$i+j = p+q$\\$i\ge p+1$}L^{ij}\right)}{\left(\myunder[]{\bigoplus}{$i+j=p+q$\\$i \ge p+1$}L^{ij} + \Im d^{p, q+1}\right)} = H_{II}^{pq}(L^{\bullet\bullet}) = E_1^{pq}
\end{align*}
\begin{align*}
E_2^{pq}=&H_I^{p}(H_{II}^q(L^{\bullet\bullet}))\\
\\
Z_0^{p+1, q-1}= \myunder[]{\bigoplus}{$i+j = p+q$\\$i\ge p+1$} L^{ij} &\qquad
Z_0^{p, q-1}= \myunder[]{\bigoplus}{$i+j = p+q-1$\\$i\ge p$} L^{ij}
\end{align*}
$K^\bullet$ -- комплекс.
%f_{*}: mod\text{-}S \to mod\text{-}R
\begin{to_thr}[Гротендик]
$\A$, $\B$, $\mathcal{C}$ -- абелевы категории. $\F \colon \A \to \B$, $\mathcal{G}\colon \B \to \mathcal{C}$ -- точные слева. $\mathcal{R}_{\A}$, $\mathcal{R}_{\B}$ -- соответствующие классы приспособленных объектов. Пусть также $\F(\mathcal{R}_{\A}) \subset \mathcal{R}_{\B}$
\begin{itemize}
    \item Тогда $\exists$ естественный изоморфизм между функторами $R(\mathcal{G}\circ \F)\cong R\mathcal{G}\circ R\mathcal{F}$.
    \item $\exists$ с. п. Гротендика $E_2^{pq} = R^p \mathcal{G} (R^q\F(X))$ $\Rightarrow$ $R^{p+q}(\F \circ \G)(X)$
\end{itemize}
\end{to_thr}
\begin{proof}
\begin{to_def}[Резольвента Картана-Эйленберга]
$K^\bullet$ -- комплекс $L^{ij}$ -- резольвента К.-Э. для $K^\bullet$:
\begin{itemize}
    \item $L^{ij}$ -- ограниченный двойной комплекс в $IV$ квадранте, $L^{ij}$ -- инъективные (приспособленные к $\F$).
    \item $\varepsilon\colon K^\bullet \to L^{\bullet, 0}$.
    \item Комплекс $0 \to K^i \to K^{i, 0} \to K^{i, 1}\to \ldots$ -- точен.
\end{itemize}
\end{to_def}
\begin{to_thr}
$ $\\
\begin{itemize}
    \item $K^\bullet$ -- имеет резолвенту К.-Э.
    \item Она определена однозначно с точностью до гомотопической эквивалентности комлексов.
    \item $\forall f\colon K^\bullet \to \tilde{K}^\bullet$ -- продолжается до морфизма резолвент однозначно с точностью до гомотопической эквивалентности. 
\end{itemize}
\end{to_thr}
\end{proof}
\begin{to_suj}[Миттаг-Леффлер]
$\lbrace p_1, \ldots, p_N\rbrace \in \mathbb{C}$. Хотим построить мероморфную функцию с фикс. главными частями рядов Лорана в точках $p_1, \ldots, p_N$. Выберем $U_i \supset p_i$ -- откр. окр. т. $p_i$. Тогда локально решение существует $f_i$ -- соответствующая мероморфная функция, решающая заадчу в $U_i$.\\
$\mathcal{O}$ -- пучок регулярных функций. $M$ -- пучок мероморфныз функций. $f_{ij} = (f_i - f_j)\vert_{U_i\bigcap U_j}$ $\in$ $\mathcal{O}(U_i\bigcap U_j)$.
\[ f_{ij} + f_{jk} + f_{ki} = 0\]
\[
g_i \in \mathcal{O}(U_i)
\]
\[f_{ij} = g_i - g_j\]
\[\tilde{f}_i = f_i - g_i\]
\end{to_suj}
Обозначим $\mathcal{PP} = \mathcal{M}/\mathcal{O}$
\bee
\begin{tikzcd}
    0 \arrow[r]& \mathcal{O} \arrow[r]& \mathcal{M} \arrow[r]& \mathcal{PP} \arrow[r]& 0
\end{tikzcd}
\eee
Пусть $\F$ -- пучок на $X$, $f\colon X \to Y$ -- непрерывное отображение. 
\begin{itemize}
    \item $\Gamma (U, \F)=\F (U)$ -- функтор глобального сечения.
    \item $\F_x = \varinjlim_{U \ni x}= \F(U)$ -- стебель пучка.
    \item $\F \oplus \G;\G \otimes \F = \lbrace U \mapsto \F(U) \otimes \G(U)\rbrace^{+}$
    \item $f_*\F(V) = \F(f^{-1}(V))$
    \item $f^{-1}\G (U) = \varinjlim_{V\ni f(U)}\G(V)$
    \item $\Gamma(X, \F) = f_* \F ; f \colon X \to \lbarce pt \rbrace$
    \item $f^{-1}$ -- точный, $f_*$ -- точный слева $\Rightarrow$ $\Gamma(X, \F)$ -- точный слева.
    \item  $f^{-1} ? f_*$
\end{itemize}
\bee
\begin{tikzcd}
    0 \arrow[r]& \Gamma(\mathcal{O}) \arrow[r]& \Gamma(\mathcal{M}) \arrow[r]& \Gamma(\mathcal{PP}) \arrow[r]& R^1\Gamma(\mathcal{O}) \arrow[r]& R^1\Gamma(\mathcal{O}) \arrow[r]& \ldots\\
     & H^0(\mathcal{O}) & & H^0(\mathcal{PP}) & H^1 (\mathcal{O})& & & 
\end{tikzcd}
\eee
\begin{to_claim}
$\F$ -- пучок на $X$. $\F \in Sh(X)$. В категории пучков достаточно много инъективных объектов (но, как правило не достаточно проективных).
\end{to_claim}
\begin{proof}
$\F_x \hookrightarrow I(x)$, $\mathcal{I}(U) = \prod_{x\in U} I(x)$ -- инъективный.\\
Проективных не достаточно много. $X$-- топологическое пространство, локально односвязно, у точек нет минимальных(?) окрестностей.
\[
i \colon\lbrace x \rbrace \hookrightarrow X
\]
$i_* \Z$ -- пучок-небоскрёр. Выберем произвольную окрестность $U(x)$, $V(x)\subset U$. Предположим существует проективное накрытие $P \twoheadrightarrow i_*\Z(U)$. Продолжение нулём $\Z_V$
\[W \mapsto \begin{cases}
    \Z, & W \subsset V \\ 0, & \text{иначе}
\end{cases}\]
\bee
\begin{tikzcd}
     & P \arrow[d]\arrow[ld]\\
    \Z_V \arrow[r, two head]& i_*\Z 
\end{tikzcd}
\qquad
\begin{tikzcd}
     & P(U) \arrow[d]\arrow[ld]\\
    \underset{=0}{\Z_V(U)} \arrow[r, two head]& i_*\Z (U)
\end{tikzcd}
\eee
\end{proof}
\begin{to_com}
В $Sh(X)$ -- достаточно много инъективных объектов $\Rightarrow$ $\exists R \Gamma \colon \mathcal{D}^+(Sh) \to \mathcal{D}^+(\Ab)$
\end{to_com}
\begin{to_def}[Когомологии Чеха с коэффициентами в пучке]
$X$ -- т.п., $R_x$ -- пучок колец, $\F$ -- пучок $R_x$-$mod$. $U = \lbrace U_{\alpha}\rbrace$ -- локально конечное покрытие.
\[C^0(U, \F) = \prod_{\alpha\in \A}\F(U_{\alpha})\]
\[C^1(U, \F) = \prod_{\alpha_0 \neq \alpha_1 }\F(U_{\alpha_0}\bigcap U_{\alpha_1})\]
\[\ldots\]
\[C^p(U, \F) = \prod_{\alpha_0 \neq \ldots \neq \alpha_p}\F(\bigcap_{i = 1\ldots p}U_{\alpha_i})\]
\[\delta\colon C^p(U, \F) \to C^{p+1}(U, \F)\]
\[C^0 \to C^1 \to C^2\]
\[(\sigma_U; \sigma_V; \sigma_W) \mapsto (\sigma_{UV};\sigma_{VW};\sigma_{WU}) \mapsto \sigma_{UVW} \]
\[\sigma_{UV} = \sigma_U - \sigma_V\vert_{U\bigcap V}\]
\[\sigma_{VW} = \sigma_V - \sigma_W\vert_{V\bigcap W}\]
\[(\delta \sigma)_{i_0\ldots i_{p}} = \sum_{j=0}^{p}(-1)^j\sigma_{i_0, \ldots, \hat{i_j}, \ldots i_{p}}\]
\[\delta^2 = 0\]
\[Z^i (U, \F)= \lbrace \sigma \in C^i(U, \F)\vert \delta\sigma = 0\rbrace\]
\[B^i (U, \F)= \lbrace \sigma \in C^i(U, \F)\vert \exists \tau \in C^{i-1} \colon \delta \tau = 0 \rbrace\]
\[\hat{H}^i = Z^i / B^i\]
$U' ? U$ -- измельчение $U$. $\forall \alpha'\in \A' \exists \alpha\in \A \colon U_{\alpha'}\subset U_{\alpha}$
\[\varphi \colon \A' \to \A\]
\[\rho_{\varphi}\colon C^p (U \F) \to C^p(U', \F)\]
\[(\rho_{\varphi}\sigma)_{i'_o\ldots i'_p} = \sigma_{\varphi(i_0')\varphi(i_1')\ldots\varphi(i_p')}\vert_{U_{i'_0\ldots i'_p}}\]
\end{to_def}
Хотя отображение $\varphi$ и определено неодназначно, оно обладает следующим свойством: Пусть есть второе отображение $\varphi$ и $\psi$ $\colon \A' \to \A$, тогда морфизмы комплексов $\rho_\varphi\sim \rho_{\psi}$ -- гомотопически эквивалентны.
\[H^p(\F) = \varinjlim_{U}\hat{H}^p(U, \F)\]
где предел берётся по всем локально конечным покрытиям.
\begin{to_thr}[Лере о покрытии]
$X = \bigsup U_i$ -- локально конечное покрытие. $H^q (\bigcap_{k = 0, \ldots p}U_{i_k}, \F) = 0$, $p\ge 0$, $q\ge 0$. $\F$ -- ацикличный пучок на пересечении. Тогда $H^p(\F) = \hat{H}^p(U, \F)$.
\end{to_thr}
\begin{to_ex}
$\Omega$ -- пучок дифференциалов на гладкой проективной кривой $\mathbb{P}^1$. $U = \lbrace \mathbb{A}^1; \mathbb{A}^1\rbrace$.\\
\[\mathbb{A}^1 = Spec\mathbb{C}[x] = U\] 
\[\mathbb{A}^1 = Spec\mathbb{C}[y] = V\]
\[x\mapsto \dfrac{1}{y}\]
\[\Gamma(U, \Omega) = \mathbf{k}[x]dx\]
\[\Gamma(V, \Omega) = \mathbf{k}[y]dy\]
\[dy \mapsto -\dfrac{1}{x^2}dx\]
\[\Gamma(V\bigcapU, \Omega) = \mathbf{k}[x, \dfrac{1}{x}]dx\]
Комплекс Чеха:
\bee
\begin{tikzcd}
    0 \arrow[r]& C^0 \arrow[r]& C^1 \arrow[r]& 0
\end{tikzcd}
\eee
\[(f(x)dx, g(y)dy)\]
\[(f(x)dx, -g(x)\dfrac{1}{x^2}dx)\]
\[f(x)dx - g(\dfrac{1}{x})\cdot \dfrac{1}{x^2}dx = 0\]
\[f = g = 0 \quad \Rightarrow \quad \hat{H}^0(U, \Omega) = 0\]
\[H^1(U, \Omega) = \mathbf{k}[x, \dfrac{1}{x}]/(f(x)-g(\dfrac{1}{x})\dfrac{1}{x^2}dx) = (x^{-1}dx)\cong \mathbf{k}\]
\end{to_ex}
\begin{to_thr}
$X$ -- локальное окольцованное пространство со структурным пучком $\mathcal{R}_x$. 
\begin{itemize}
    \item[a] $\Phi\colon \mathcal{R}_x\text{-}mod \to Sh{Ab}$ -- забывающий функтор. Тогда $\exists$ канонический изоморфизм $R\Gamma \cong R(\Gamma\circ \Phi)$.
    \item[b] Th. Лере, $H^i(X ,\F)$ совпадает с когомологиями Чеха. 
    \item[c] $H^i(X, \F) = Ext^i_{\mathcal{R}_x}(\mathcal{R}_x, \F)$
    \item[d] $R^i f_*(\F)\cong (U \to H^i(f^{-1}(U);\F))^+$
    \item[e] Спектральная последовательность Лере $X \overset{f}{\to} Y \overset{g}{\to} Z$, $\F$ -- пучок на $X$. 
    \[E^{pq}_r = R^p g_*(R^qf_*(\F))\Rightarrow R^{p+q}(gf)_*(\F)\]
    \item[f] Спектральная последовательность Чеха $U$-покрытие.
    \[E_r^{pq} = \hat{H}^p(U, \mathcal{H}^q(\F))\Rightarrow H^n(X, \F)\]
    \[\mathcal{H}^q(\F)\text{ -- пучко когомологий $\F$}\]
    \[\lbrace U \to H^q(U, \F)\rbrace^+\]
\end{itemize}
\end{to_thr}
\begin{to_def}
Пучок $\F$ назывется вялым, если все морфизмы сужения сюръективны.\footnote{Можно продолжать на большие множества, поднимать морфизмы сужения}
\end{to_def}
\begin{to_claim}
Вялых пучков достаточно много $\G(U) = \prod_{x \in U} \F_x$ -- очевидно вялый. $\F \hookrightarrow \G$
\end{to_claim}
\begin{to_claim}
Инъективный $\Rightarrow$ вялый. $\mathcal{I} \hookrightarrow \G$ $\Rightarrow$ $\mathcal{I}$ -- прямое слагаемо в $\G$ $\Rightarrow$ вялый.
\end{to_claim}
\begin{to_claim}
Пусть дана точная последовательность пучков:
\bee
\begin{tikzcd}
    0 \arrow[r]& \F \arrow[r, "\varphi"]& \G \arrow[r, "\psi"]& \mathcal{H} \arrow[r]& 0
\end{tikzcd}
\eee
$\F$ -- вялый. Тогда:
\bee
\begin{tikzcd}
    0 \arrow[r]& \Gamma(X, \F) \arrow[r, "\varphi"]& \Gamma(X, \G) \arrow[r, "\psi"]& \Gamma(X, \mathcal{H}) \arrow[r]& 0
\end{tikzcd}
\eee
\end{to_claim}
\begin{proof}
    $s \in \Gamma(\mathcal{H})$, $E = \lbarce (U, t)| U \subset X; t \in \G (U)\colon \psi(t) = s|_U\rbrace$.
    \[(U', t')\prec (U'', t'') \Longleftrightarrow U' \subset U'', t''|_{U'} = t'\]
    По лемме Цорна найдётся максимальный элемент $U, t$. Предположим, что $U \neq X$, $\in X/U$; $V(x)$ -- окрестность $\exists t_1 \in \G(V)\colon s|_V = \psi(t_1)$
    $(t-t_1)|_{U\bigcap V} = \varphi(r_{UV})$, $r_{UV}\in \F(U\bigcap V)$ $\Rightarrow$ $r \in \Gamma(\F)$: $r|_{U \bigcap V} = r_{UV}$, $t_2 = t_1 + r|_{V}$, $t_1$; $t_2$ -- согласованы $\Rightarrow$ противоречие.\\
    $\G$, $\mathcal{H}$ -- вялы $\Rightarrow$ $\F$ -- вял.
    \[s \in \mathcal{H}(U) \qquad t', t \in \G(U) \qquad r = t- t'\]
    \[\tilde{s}\in \Gamma(U)\]
    \[t+r \in \Gamma(\G)\]
    \[\psi(t) = \tilde{s} \qquad \psi(t') = \tilde{s} \qquad \Rightarrow \qquad (\tilde{r}) = 0 \qquad \Rightarrow \qquad \tilde{r}\in \Gamma(\F)\]
    $\Gamma$ преводит ацикличные комплексы (огр. слева) вялых пучков в ацикличные 
    \bee
\begin{tikzcd}
    0 \arrow[r]& \F^0 \arrow[r]& \F^1 \arrow[r]& \ldots
\end{tikzcd}
    \eee
    \bee
\begin{tikzcd}
    0 \arrow[r] & Z^i(\F^\bullet) \arrow[r]& \F^i \arrow[r]& Z^{i+1}(\F^\bullet)
\end{tikzcd}
    \eee
    \bee
\begin{tikzcd}
    0 \arrow[r] & \F^0 \arrow[r]& \B^i(\F^\bullet) \arrow[r]& 0
\end{tikzcd}
    \eee
    {\color{red}************} $\lbrace U_\alpha \rbrace$ $I = (i_0, \ldots, i_p)$
    \[j_I\colon U_{i_0}\bigcap\ldots\bigcap U_{i_p} \hookrightarrow X\]
    \[\F_I = j_* j^* \F\]
    \[H^q (U_{i_0}\bigcap \ldots U_{i_p}, j_I_* j_I^*\F) = 0 \text{, при } p\ge 0, q\ge 0\]
    \[0 \to \F \to \mathcal{K}^0 \to \mathcal{K}^1 \to \ldots \text{ -- вялая мягкая резольвента.}\]
    \[C^p(\K^q) \text{ -- вялые пучки.}\]
    $e^p(\K^q)$ -- двойной комплекс. 
    \[E^{pq}_r = H^p(H^q(U; C^\bullet(\F)))\qquad \Rightarrow \qquad H^{p+q}(\Gamma(X, C^\bullet(\F)))\]
    \[Ext^i(\mathcal{R}_x; \F) = R^i Hom(\mathcal{R}_x; \F) = R^i\Gamma(\F) = H^i(\F)\]
    \[\F \hookrightarrow I^\bullet \qquad R^q f_* (\F) = H^q f_*(I^\bullet)\]
    \[(U \to H^q(U; f_* I))^+\]
\end{proof}
\end{document}