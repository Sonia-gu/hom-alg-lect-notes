\documentclass[../main.tex]{subfiles}
\begin{document}
\section{Классический производный функтор и класс приспособленных объектов}
Пусть $\A$, $\B$ -- абелевы категории.
 \begin{to_def}
Функтор $H: \D{A}\rightarrow\B$ называется когомологическим, если выделенные треугольники \begin{tikzcd}[sep=small]X \arrow[r]& Y \arrow[r]& Z \arrow[r]& X[1] \end{tikzcd} он переводит в длинные точные последовательности
\begin{tikzcd}[sep=small]\ldots \arrow[r] & H(X[i]) \arrow[r]& H(Y[i]) \arrow[r]& H(Z[i]) \arrow[r]& H(X[i+1])\arrow[r]&\ldots \end{tikzcd}
\end{to_def}
\begin{to_ex}
$ $
\begin{itemize}
    \item $H^0$ -- когомологический\footnote{snake lemma}
    \item $\Hom(X, -)$ -- тоже.
\end{itemize}
\end{to_ex}
\begin{to_def}
 \label{classicDerFunc}
Классический производный функтор 
\begin{align*}
    \Ln{i}{F} = H^{-i}(\DlF{F})\\
    \Rn{i}{F} = H^{i}(\DlF{F})\\
    \Ln{i}{F}, \Rn{i}{F}: \A \rightarrow \B
\end{align*}
\bee
\begin{tikzcd}
\A \arrow[r, "Q_{\A}^-"]\arrow[rrr, bend right, "\Rn{n}{F}" '] & \Dr{A} \arrow[r, "\DrF{F}"] & \Dr{B} \arrow[r, "H^0\circ (n)" ] & \mathcal{B}
\end{tikzcd}
\eee
\end{to_def}
\begin{to_claim}
Пусть $\F$ -- точный слева функтор. \begin{tikzcd}[sep=small]0 \arrow[r] & X \arrow[r]& Y \arrow[r]& Z \arrow[r]& 0 \end{tikzcd} -- короткая точная последовательность. Тогда $\exists$ длинная точная последовательность вида:
\bee
\begin{tikzcd}[sep=small]
    0 \arrow[r] &\F(X) \arrow[r] &\F(Y)\arrow[r] &\F(Z)\arrow[r] &\Rn{1}{F}(X) \arrow[r] &\Rn{1}{F}(Y)\arrow[r] &\Rn{1}{F}(Z) \arrow[r]& \ldots  
\end{tikzcd}
\eee
\end{to_claim}
\begin{proof}
Действуем по определению производного функтора. Для этого короткую точную последовательность мы погружаем в производную категорию и выбираем квазиизоморфные им комплексы с приспособленными членами. Далее мы подействуем почленно производным функтором и сделаем из короткой точной последовательности треугольник. Для этого в производной категории рассмотрим морфизм комплексов с когомологиями, сосредоточенными в нулевом члене\footnote{строго полное вложение в производную категорию}. Конусом данного морфизма будет являться комлекс с нулевой когомологией $C(f) \overset{qis}{\cong} Z[0]$. 
\bee
\begin{tikzcd}[sep=small]
                    &                & 0                &\\
X[0] \arrow[r, "f"] & Y[0] \arrow[r] & Y \arrow[u]      &C(f)\cong Z[0]\\
                    &                & X \arrow[u, "f"'] &\\
                    &                & 0 \arrow[u]      &\\
\end{tikzcd}
\eee
Для каждого комплекса мы находим его резольвенту и заменяем исходный треугольник треугольником соответствующих резольвент.
\bee
\begin{tikzcd}[sep=small]
    0\arrow[r]&\Rn{0}{}_X\arrow[r]&\Rn{1}{}_X\arrow[r]&\ldots = \Rn{\bullet}{}_X\\
    0\arrow[r]&\Rn{0}{}_Y\arrow[r]&\Rn{1}{}_Y\arrow[r]&\ldots = \Rn{\bullet}{}_Y\\
    0\arrow[r]&\Rn{0}{}_Z\arrow[r]&\Rn{1}{}_Z\arrow[r]&\ldots = \Rn{\bullet}{}_Z
\end{tikzcd}
\eee
Так как производный функтор точен для класса приспособленных объектов, мы получим выделенный треугольник после его применения к выделенному треугольнику, полученному на предыдущем шаге.
\bee
\begin{tikzcd}[sep=small]
    \Rn{\bullet}{}_X \arrow[r]& \Rn{\bullet}{}_Y \arrow[r]&\Rn{\bullet}{}_Z \arrow[r]&\Rn{\bullet}{}_Z[1]\\
    \KlF{F}\Rn{\bullet}{F}_X \arrow[r]& \KlF{F}\Rn{\bullet}{}_Y \arrow[r]&\KlF{F}\Rn{\bullet}{}_Z \arrow[r]&\KlF{F}\Rn{\bullet}{}_Z[1]\\
\end{tikzcd}
\eee
 Когомологический функтор $H^0$ сделает из выделенного треугольника длинную точную последовательность когомологий:
 \bee
\begin{tikzcd}[sep=small]
    0 \arrow[r] &H^0\DlF{F}\Rn{\bullet}{}_X \arrow[r] &H^0\DlF{F}\Rn{\bullet}{}_Y \arrow[r] &H^0\DlF{F}\Rn{\bullet}{}_Z\arrow[r] &H^0\circ[1](\DlF{F}\Rn{\bullet}{}_X) \arrow[r] & \ldots  
\end{tikzcd}
\eee
\end{proof}
\begin{to_def}
Объект $X$ называется $\F$--ацикличным, если $\Rn{n}{F}(X) = 0$ $\forall n\neq 0$.
\end{to_def}
Обозначим $\mathcal{Z}$ класс $\F$-- ацикличных объектов. До сих пор мы получали существование производного функтора из наличия приспособленного класса. Имеет место следующее частичное обращение этого рассуждения:
\begin{to_claim}
$\exists$ класс приспособленных к $\F$ объектов $\Leftrightarrow$
    $\exists$ достаточно большой\footnote{$\forall$ $X \in \A$ является подобъектом $\F$--ацикличного (если $\F$-- точен слева), или факторобъектом ацикличного (если $\F$-- точен справа)} $\mathcal{Z}$.
\end{to_claim}
\begin{proof} Проведём доказательство для точного слева функтора $\F$.\\
$\Leftarrow$ Пусть $\mathcal{R}$-- класс приспособленных к $\F$ объектов. Тогда $\DF{F}(X[0])\overset{qis}{\cong} \F(X)[0]$ $\forall X \in \mathcal{R}$, поэтому $\mathcal{R} \subset \mathcal{Z}$ и $\mathcal{Z}$-- достаточно большой, так как $\mathcal{R}$-- достаточно большой.\\
$\Rightarrow$ Пусть теперь $\mathcal{R} \subset \mathcal{Z}$ -- достаточно большой подкласс $\mathcal{F}$-ацикличных объектов. Чтобы установить приспособленность достаточно показать, что $\F$ переводит ацикличные комплексы из $Kom^{\pm}(\mathcal{R})$ в ацикличные. Если мы имеем ацикличную тройку вида \begin{tikzcd}[sep=small] 0 \arrow[r]& K^0 \arrow[r] & K^1 \arrow[r] & K^2 \arrow[r] & 0 \end{tikzcd}, то точность \begin{tikzcd}[sep=small] 0 \arrow[r]& \F(K^0) \arrow[r] & \F(K^1) \arrow[r] & \F(K^2) \arrow[r] & 0 \end{tikzcd} следует из $\mathcal{R}^1\F(K^0)=0$. В общем случае можно отщиплять точные тройки следующим образом:
\bee
\begin{tikzcd}[sep=small]
    0 \arrow[r]& K^0 \arrow[r, "d^0"] = X^0& K^1 \arrow[r, "d^1"]\arrow[rd]& K2 \arrow[r]& \ldots\\
               &               &                            & im d^1 \arrow[u]=X^1       & \\
               &               &                            & X^i = im d^i &
\end{tikzcd}
\eee
Далее так как $X^i, K^{i+1} \in \mathcal{Z}$ $\Rightarrow$ $X^i \in \mathcal{Z}$ $\Rightarrow$ \begin{tikzcd}[sep=small] 0 \arrow[r]& \F(X^i) \arrow[r]& \F(K^{i+1})\arrow[r]& \F(X^{i+1}) \arrow[r] & 0 \end{tikzcd} -- точны $\Rightarrow$ $\F(K^\bullet)$ -- ацикличен.
\end{proof}
\begin{to_claim}
$\forall$ достаточно большой $\mathcal{Z}$ приспособлен к $\F$\footnote{любой класс приспособленных лежит в достаточно большом $\mathcal{Z}$}
\end{to_claim}
\begin{to_claim}
В достаточно большом $\mathcal{Z}$ лежат все инъективные\footnote{точный справа} и проективные \footnote{точный слева} объекты категории $\A$.
\end{to_claim}
\end{document}