\documentclass[../main.tex]{subfiles}
\begin{document}
\section{Функтор $Tor$}
\begin{to_note}
Если в $\A$ достаточно много проективных объектов и $\gldim \A = n$, а $\F$ -- точный справа функтор, то 
\[\forall m > n\quad
L^m\F{(X)} = 0\quad
\forall X \in \A\]
\end{to_note}
\begin{to_note}
Комплекс де-Рама является резольвентой постоянного пучка на гладком многообразии.
\end{to_note}
\begin{to_claim}
Пусть в $\A\in \Ab$ имеем комплекс $K^\bullet$ с двумя нетривиальными соседними когомологиями.
\[
H^n(K^\bullet) = \begin{cases} H^0, &n = 0\\ H^1, &n = 1\\ 0, &иначе\end{cases}
\]
Комплекс такого вида классифицируется с помощью $H^0$, $H^1$ и $Ext^2(H^1, H^0)$ с точностью до $qis$.
\end{to_claim}\\
\textbf{Hint:} Под классификацией понимаются соответствующие классы эквивалентности квазиизоморфных комплексов в производной категории. Комплекс в производной категории будет квазиизоморфен комплексу с двумя нетривиальными членами\footnote{строго-полное вложение $\A$ в $\D{A}$, обрезание комплексов} \begin{tikzcd}[sep=small] 0 \arrow[r]& K^0 \arrow[r]& K^1 \arrow[r]& 0 \end{tikzcd}. \\
Такой комплекс достаивается до комплекса \begin{tikzcd}[sep=small] H^0 \arrow[r]& K^0 \arrow[r]& K^1 \arrow[r]& H^1\end{tikzcd} $\in$ $Ext^2(H^1, H^0)$. \\
Любой $qis$ задаёт эквивалентность расширений.
\bee
\begin{tikzcd}[sep=small]
0 \arrow[r]& K^0 \arrow[r]\arrow[d, "\sim"]& K^1 \arrow[r]\arrow[d, "\sim"]& 0\\
0 \arrow[r]& H^0 \arrow[r]& H^1 \arrow[r]& 0
\end{tikzcd}
\eee
\bee
\begin{tikzcd}[sep=small]
0 \arrow[r]& H^0 \arrow[r]\arrow[d, "\sim"]& K^0 \arrow[r]\arrow[d]& K^1 \arrow[r]\arrow[d]& H^1 \arrow[r]\arrow[d, "\sim"]& 0\\
0 \arrow[r]& H^0 \arrow[r]& L^0 \arrow[r]& L^1 \arrow[r]& H^1 \arrow[r]& 0
\end{tikzcd}
\eee
Но также любая эквивалентность расширений будет давать квазиизоморфизм комплексов. Эквивалентность расширений задаётся последовательностью домиков, определяющую квазиизоморфизм комплексов.
\bee
\begin{tikzcd}
        &                               &K^\bullet\arrow[ld]\arrow[rd]\arrow[lldd, bend right]\arrow[rrdd, bend left]& & & &\\
        &K^\bullet\arrow[ld]\arrow[rrrd]& & L^\bullet\arrow[rd]\arrow[llld]& \\
    H^0[0] & & & & H^1[1]
\end{tikzcd}
\eee
\begin{to_def}
\[
Tor_n^R (A, -) = L_n(A\otimes)(-)\footnote{$A\otimes(-)$-- сопряженный слева к фуктору $Hom$, а значит он точен справа, значит у него есть левые производные}
\]
\end{to_def}
\[
Ext^\bullet(A, -)\colon \Db{A} \to \Db{A} \vdash Tor^\bullet(A, -)\colon \Db{A} \to \Db{A}
\]
\begin{to_ex}
$\A \in \Ab$
\[
Tor_n (\Z_p, A)
\]
Для точного справа функтора выпишем проективную резольвенту 
\bee
\begin{tikzcd}[sep=small]
    0\arrow[r]& \Z \arrow[r, "\cdot p"]& \Z \arrow[r, "\varepsilon"]& \Z_p \arrow[r]& 0\\
              &                       & 0 \arrow[u, dashed]         &               & \\  
\end{tikzcd}
\eee
Далее применим к ней ковариантный фунтктор $\underset{\Z}{\otimes}A(-)$ и вычислим когомологии
\bee
\begin{tikzcd}[sep=small]
    A \arrow[r, "\cdot p"]& A \\
                          & 0 \arrow[u, dashed]  
\end{tikzcd}
\eee
\begin{align*}
    Tor_0(\Z/ p\Z; A) &= \Z / p\Z \underset{\Z}{\otimes} A = A / pA\\
    Tor_1 (\Z / p\Z; A) &= \{ a \in A | pa = 0\} \text{ -- элементы порядка } p
\end{align*}
\end{to_ex}
\begin{to_ex}
Пусть $A$ -- конечно порождена, то есть имеет структуру $A = (\underset{k}{\oplus}\Z^k) \oplus (\underset{i=1, N}{\oplus}\Z / p_i\Z)$. Тогда $Tor$ также будет раскладываться в сумму \\
\begin{align*}
    Tor_1 (A; B) = \underset{i=1, N}{\oplus} Tor_1(\Z / p_i\Z; B)
\end{align*}
\end{to_ex}
\begin{to_ex}
Если же $A$-- произвольная группа, то она является пределом конечнопорждённых груп $A = \underrightarrow{\lim} A_i$, а также всегда представима в виде прямой суммы свободной группы и группы кручения $A \cong \mathcal{T}A\oplus\F A$ $\Rightarrow$ если $A$ -- без кручения, то $Tor(A; B)=0$.
\end{to_ex}
\begin{to_def}
$A$ -- \textbf{плоский} $\Leftrightarrow$ $Tor_1(A; B) = 0$, $\forall$ $B$.
\end{to_def}
Если $A$ -- плоский, то функтор $A\otimes(-)$-- точный.
\begin{to_claim}
Плоские модули -- приспособлены к функтору $(-)\otimes B$.
\end{to_claim}
\begin{to_claim}\label{projisflat}
Проективный модуль $\Rightarrow$ плоский.
\end{to_claim}
\begin{to_claim}
Если $\mathcal{R}$ -- \textbf{PID}\footnote{КГИ} $\Rightarrow$ (плоский) $\Leftrightarrow$ (без кручения).
\end{to_claim}
\begin{to_claim}
\[A \in \Ab\]
\[Tor_1(\Q / \Z; A)= \mathcal{T}A\text{ -- кручение $A$}\] 
\[\underrightarrow{\lim}\text{ }\Z/p\Z = \Q / \Z\]\footnote{при подстановке в $Tor$ получаем прямой предел подгрупп элементов имеющий заднаный порядок для всех возможных порядков, то есть кручение группы. $\underrightarrow{\lim}$ учитывает пересечения всех таких подгрупп}
\end{to_claim}
\begin{to_claim}[$\mathcal{T}=0$, $Tor \neq 0$]
Рассмотрим пример, когда модуль без кручения над кольцом $\mathcal{R}\neq$ \textbf{PID}, имеет ненулевые $Tor$. Стандартным примером не \textbf{PID} является $\mathcal{R} =\mathbf{k}\left[x, y\right]$. Рассмотрим в нём модуль $A = (x; y) \neq$ \textbf{PI}. Напишем резольвенту Кошуля
\bee
\begin{tikzcd}
    0 \arrow[r]& \mathcal{R} \arrow[r, "{\begin{pmatrix} -y \\ x \end{pmatrix}}"]& \mathcal{R}\oplus \mathcal{R} \arrow[r, "{\begin{pmatrix}x & $ $ y \end{pmatrix}}"]& \mathbf{k} \arrow[r]& 0\\
    0 \arrow[r]& \mathcal{R} \arrow[r, "{\begin{pmatrix} -y \\ x \end{pmatrix}}"]& \mathcal{R}\oplus \mathcal{R} \arrow[r, "{\begin{pmatrix}x & $ $ y \end{pmatrix}}"]& (x; y) \arrow[r]& 0
\end{tikzcd}
\eee
Применяя функтор $\underset{k[x; y]}{\otimes}\mathbf{k} (-)$, получаем комплекс с нулевыми морфизмами:\\
\bee
\begin{tikzcd}[sep=small]
    0 \arrow[r]& \mathbf{k} \arrow[r, "0"]& \mathbf{k} \oplus \mathbf{k} \arrow[r, "0"]& \mathbf{k} \arrow[r]& 0 \end{tikzcd}
\eee
\[
Tor_1(A, \mathbf{k}) =\mathbf{k} \oplus \mathbf{k} \neq 0
\]
\end{to_claim}
\begin{to_ex}[кольцо с делителями нуля]
Пусть $\mathcal{R} = \Z_m$, $M = \Z_d$ $\in\mathcal{R}-mod$, $d|m$. 
\bee
\begin{tikzcd}[sep=small]
    \ldots\arrow[r]& \Z_m \arrow[r, "\cdot d"]& \Z_m \arrow[r, "m/d"]& \Z_d \arrow[r]& 0
\end{tikzcd}
\eee
Мы получили бесконечную циклическую проективную резольвенту.\footnote{Наличие делителей нуля тесно связано с бесконечной глобальной размерностью.} Применим к ней $\underset{\Z_m}{\otimes}B (-)$:
\bee
\begin{tikzcd}[sep=small]
    \ldots\arrow[r, "\cdot d"]& B \arrow[r, "\cdot m/d"]& B \arrow[r, "\cdot d"]& B \arrow[r]& 0
\end{tikzcd}
\eee
Теперь вычислим когомологии
\begin{align*}
    Tor_0(M; B) &= B / dB\\
    Tor_{2k+1}(M; B) &=  \lbrace b \text{ }| ord(b)=d \rbrace \text{ }/ (m/d)  \\
    Tor_{2k}(M; B) &=  \lbrace b \text{ }| ord(b)= m/d \rbrace \text{ }/ (d)
\end{align*}
\end{to_ex}
\end{document}