\documentclass[../main.tex]{subfiles}
\begin{document}
\section{Спектральные последовательности}
Работаем, например, в категории $Kom(A)$. Будем говорить, что на объекте $A$ задана убывающая регулярная фильтрация, то есть цепочка вложенных друг в друга подобъектов $A \supset \ldots F^p A \supset F^{p+1}A \supset \ldots$, регулярность означает, что:
\begin{itemize}
    \item $\bigcap F^p A = 0$ \item  $\bigcup F^p A = A$
\end{itemize}
Тогда по такой последовательности можно построить \textbf{градуировочный фактор} $E^p = F^p A / F^{p+1} A$.
"Пристёгивание" факторов к подмодулю будем обозначать как $F^N \sqsupset E^{N-1} \sqsupset E^{N-2}\sqsupset\ldots$.
\textbf{Вопрос:} если известны все градуировочные факторы фильтрации, можем ли мы восстановить наш исходный объект?
\begin{to_def}
Спектральной последовательностью является набор данных, состоящий из 
\begin{itemize}
\item стопки листов с занумерованными клетками, в которых находятся объекты категории\footnote{$r$ -- номер листа, а $pq$ -- номер клетки}
\item дифференциала между объектами листа \footnote{бьёт обобщённым "ходом коня", на нудевом шаге он бьёт вправо, потом вверх, а потом всегда попадает на соседнюю диагональ}
\item изоморфизма между когомологиями и следующим листом\footnote{на каждом следующем листе стоят когомологии предыдущего, дифференциалы на каждом следующем листе индуцированы}
\item изоморфизма между пределом когомологий на трансфинитном листе и градуировочными факторами фильтрации
\item объект на котором задана фильтрация 
\end{itemize}
\[(E^{pq}_r, E^n, d^{pq}_r, \alpha^{pq}_r, \beta^{pq})\]
\[d^{pq}_r&\colon E^{pq}_r\to E^{p+r, q-r+1}_r  \]
\begin{align*}
    \alpha_r^{pq}\colon H^{pq}_r(E_r^{\bullet\bullet}) \to E^{pq}_{r+1}
\end{align*}
Начиная с некоторого листа для любого члена все дифференциалы, которые бьют из него и в него зануляются, то есть
\begin{align*}
\forall (p, q) \quad \exists r_0\colon \forall r\ge r_0 \quad \hookrightarrow  \quad d^{pq}_r = 0 \quad d_r^{p-r, q+r-1}=0     
\end{align*}
Это означает, что когомологии с этого момента перестают меняться, а последовательность стабилизируется.\\
$E^n$ -- это комплекс, на котором задана убывающая регулярная фильтрация $\ldots \supset \ldots F^p E^n \supset F^{p+1} E^n \supset \ldots$.
\begin{align*}
    E^{pq}_{r_0}&\cong E_{\infty}^{pq}\\
    \beta^{pq}&\colon E_{\infty}^{pq} \to F^{p} E^{p+q} / F^{p+1} E^{p+q}
\end{align*}
\textbf{Спектральная последовательность сходится к градуировочным факторам фильтрации.}
\end{to_def}
Гротендик придумал спектральные категории, чтобы ввести спектральные последовательности.\\
Знание градуировочных факторов фильтрации не позволяет восстановить объект. Однозначности восстановления нет. 
\textbf{filtration respects differentiall structure}
\begin{to_ex}[Эйлерова характеристика]
$\chi \colon \Ab \to G$, $G$ -- абелева группа. 
Фильтрация длины 1
\bee
\begin{tikzcd}[sep=small]
0 \arrow[r]& A \arrow[r] & B \arrow[r] &C \arrow[r] & 0
\end{tikzcd}
\qquad \chi(B) = \chi(A) + \chi(C)
\eee
\[
\chi (K^\bullet) = \sum_{}^{}(-1)^n \chi (K^\bullet) = \sum (-1)^n \chi (H^n(K^\bullet))
\]
$\chi (E^n) = ?$\\
$E^{pq}_r \Rightarrow E^n$
Вычислим альтернированную сумму на листе $r$
\[E_{pq}^{\bullet} = \underset{p+q = \bullet}{\oplus} E_r^{pq}\]
\[\chi (E^{\bullet \bullet}_r) = \sum (-1)^k \chi(H^k(E_r^{\bullet})) = \chi(E_{r+1}^{\bullet}) =\ldots = \chi(E_{\infty}^{\bullet})\]
\[\chi(E^n) = \sum \chi (F^pE^n / F^{p+1}E^n) = \chi(E^{\bullet \bullet}_{\infty})\]
\end{to_ex}
\subsection{Фильтрованный комплекс}
Пусть $E^n$  -- комплекс, на котором задана фильтрация. Отметим, что дифференциал \textit{"не понижает градус фильтрации"}, т. е.  $d(F^pE^n)\subset F^p E^{n+1}$. Можем рассмотреть два варианта фильтрации:
\begin{itemize}
\item \textit{Глупая фильтрация}
\[
\widetilde{F}_p E^n = \begin{cases} 0, &n < p \\ E^n, & n \ge p \end{cases}
\]
\[
H^n(\widetilde{F}^p E^n) = \begin{cases} 0, &n < p\\ \ker d^p, & n=p \\ E^n, & n > p \end{cases}
\]
\bee
\begin{tikzcd}[sep=small]
\ldots \arrow[r]\arrow[d]& 0 \arrow[r]\arrow[d]& 0\arrow[r]\arrow[d]& E^{n+1} \arrow[r] \arrow[d]&\ldots\\
\ldots \arrow[r]\arrow[d]& E^{n-1} \arrow[r]\arrow[d]& E^{n}\arrow[r]\arrow[d]& E^{n+1} \arrow[r] \arrow[d]&\ldots\\
\ldots \arrow[r]& E^{n-1} \arrow[r]& E^{n}\arrow[r]& 0 \arrow[r]&\ldots\\
\end{tikzcd}
\eee
\item \textit{Каноническая фльтрация}
\[
F_p E^n = \begin{cases} E^n, &n < -p \\ \ker d^p, & n = -p \\ 0, & n>-p \end{cases}
\]
\[
H^n (F^p E^\bullet) \begin{cases} 0, &n > -p \\ H^n(E^\bullet), & n\le-p \end{cases}
\]
\bee
\begin{tikzcd}[sep=small]
\ldots \arrow[r]\arrow[d]& E^{-p-1} \arrow[r]\arrow[d]& E^{-p}\arrow[r]\arrow[d]& E^{-p+1} \arrow[r]\arrow[d]& \ldots\\
\ldots \arrow[r]& E^{-p-1} \arrow[r]\arrow[u]& \ker d^{-p}\arrow[r]& 0 \arrow[r]&\ldots\\
\end{tikzcd}
\eee
\end{itemize}
\begin{to_ex}[Cпектральная последовательность фильтрованного комплекса $K^\bullet$]
Для фильтрованного комплекса существует спектральная последовательность. 
\end{to_ex}
Определим следующие группы элементов комплекса: \ref{eq:cicle} группа элементов, лежащих в $F^{p}K^{p+q}}$ члене фильтрации, у которых дифференциал углубляет фильтрационный номер не более чем на $r$,  \ref{eq:cicle2} элементы, у которых номер фильтрации углубляется более чем на $r$ при применении дифференциала и границы \ref{eq:bound}. C помощью этих элементов мы определим элементы спектральной последовательности по формуле \ref{eq:specfiltr}, нулевой лист спектральной последовательности буде иметь вид \ref{eq:zero} {\color{red} возможно надо $Z$ заменить на $G$, например}
\be\label{eq:cicle}
Z_r^{pq} = d^{-1}(F^{p+r}K^{p+q+1})\bigcap F^p K^{p+q}
\ee
\be\label{eq:cicle2}
Z^{p+1; q-1}_{r-1} = d^{-1}(F^{p+r}K^{p+q+1})\bigcap F^{p+1}K^{p+q}
\ee
\be\label{eq:bound}
d (Z_{r-1}^{p-r+1, q+r-2}) = d( F^{p-r+1}K^{p+q+1})\bigcap F^p K^{p+q+1}
\ee
\be\label{eq:specfiltr}
E^{pq}_r = \dfrac{Z_r^{pq}}{Z^{p+1, q-1}_{r-1} + dZ_{r-1}^{p-r+1, q+r-2}}
\ee
\be\label{eq:zero}
E_0^{pq} = F^p K^{p+q} / F^{p+1}K^{p+q}
\ee
Утверждается, что спектральная последовательность, заданная элементами вида  \ref{eq:specfiltr} будет сходиться к градуировочным факторам фильтрации комплекса. 
\newpage
Чтобы это проверить нужно установить следующее:
\begin{itemize}
    \item Дифференциал корректно определён\footnote{Дифференциалы на когомологиях спектральной последовательности не могут быть восстановлены по начальным листам, на которых дифференциал унаследован из исходной фильтрации. Если известны первые несколько листов спектральной последовательности, то могут быть построены члены следующих листов, но не их дифференциалы.}
    \item Существую изоморфизмы между когомологиями на соседних листах
    \be\label{eq:Z}
   	\dfrac{Z^{pq}_{r+1} + Z^{p+1, q-1}_{r-1}}{Z_{r-1}^{p+1, q-1} + dZ_{r-1}^{p-r+1, q+r-2}}\quad\to\quad \mathcal{Z}(E_r^{pq})
\ee
\be\label{eq:B}
    \dfrac{dZ^{p-r, q+r-1}_{r} + Z^{p+1, q-1}_{r-1}}{Z_{r-1}^{p+1, q-1} + dZ_{r-1}^{p-r+1, q+r-2}}\quad\to\quad \mathcal{B}(E_r^{pq})
    \ee
    \item Группы корректно определены, то есть 
    \[\mathcal{Z}(E_r^{pq})/\mathcal{B}(E_r^{pq}) = E^{pq}_{r+1}\]
\end{itemize}
Проверка происходит руками и с большим количеством индексов. Поехали...
\begin{itemize}
\item[1]\textit{Определение дифференциала}
\[d_r^{pq}\colon E_r^{pq} \to E_r^{p+r, q-r+1}\]
\[dZ_r^{pq}\subset Z_r^{p+r, q-r+1}\]
\item[2]\textit{Изоморфизм между когомологиями}
\ref{eq:Z} и \ref{eq:B} являются мономорфизмами, так как в числителе стоят подгруппы $Z^{pq}_r$, а фактор берётся по одним и тем же подгруппам. Теперь выпишем явно циклы. В них будут те элементы, по которым берётся фактор на следующем шаге, отфакторизованный по подгруппам из предыдущего шага 
\[
\mathcal{Z}(E^{pq}_r)=\dfrac{Z_r^{pq}\overset{\mathbf{(2)}}{\bigcap} d^{-1}\left(Z_{r-1}^{p+r+1, q-2}+dZ_{r-1}^{p+1, q-1}\right)^\mathbf{(1)}}{Z_{r-1}^{p+r+1, q-r} + dZ_{r-1}^{p-r+1, q+r-2}}
\]
Выполняем "по действиям":
\begin{align*}
\mathbf{(1)} \quad& d^{-1}\left(Z_{r-1}^{p+r+1, q-r}+dZ_{r-1}^{p+1, q-1}\right) = d^{-1}\left(Z_{r-1}^{p+r+1, q-r})+Z_{r-1}^{p+1, q-1}\right)\\
\mathbf{(2)} \quad& Z_r^{pq}\bigcap \left(d^{-1}Z_{r-1}^{p+r+1, q-r}+Z_{r-1}^{p+1, q-1}\right) = Z_{r-1}^{p+1, q-1}+Z_r^{pq}\bigcapd^{-1}Z_{r-1}^{p+1, q-1}
\end{align*}
\[
Z_r^{pq}\bigcap d^{-1} (Z_{r-1}^{p+r+1, q-r}) = d^{-1}\left(F^{p+r}K^{p+q+1}\bigcap d^{-1}(Z_{r-1}^{p+r+1, q-r})\right)\bigcap F^p K^{p+q} \subset d^{-1}\left(F^{p+r+1}K^{p+q+1}\right)\bigcap F^pK^{p+q}\subset Z_{r+1}^{pq}
\]
\[
\alpha_r^{pq}\colon H(E_r^{pq})\overset{\conq}{\to} E_{r+1}^{pq}
\]
\[
E_{r+1}^{pq} = \dfrac{Z^{pq}_{r+1}}{Z_r^{p+1, q-1}+dZ_r^{p-r, q+r-1}}
\]
\textcolor{red}{ тут не успела за исправлениями 
\[
\dfrac{Z_{r+1}^{pq}+Z_{r-1}^{p+1, q-1}}{Z_{r-1}^{p+1, q-1}+dZ_r^{p-r, q+r-1}}= \dfrac{Z_{r+1}^{pq}}{\underset{Z_r^{p+1, q-1}}{Z_{r+1}^{pq}\bigcap(dZ_{r+1}^{p-r, q+r-1}+Z_{r-1}^{p+1, q})}} = \dfrac{Z_{r+1}^{pq}}{Z_r^{p+1, q-1}+dZ_r^{p-r, }}
\]
}
\end{itemize}
Если фильтрация конечна на каждом $K^n$, то спектральная последовательность сходится 
\[
E_{\infty}^{pq} = \dfrac{Z_r^{pq}}{Z_{r-1}^{p+1, q-1}} \quad \text{при } r > r_0
\]
\[
Z_r^{pq} = Z(F^p K^{p+q})
\]
\[
Z_{r-1}^{p+1, q-1} = Z(F^{p+1}K^{p+q})
\]
\[
\underset{p+q=r}{\oplus}E_{\infty}^{pq} = \oplus \mathbf{Gr} H^n (K^\bullet)
\]
Спектральная последовательность сходится к градуировочным факторам когомологий 
\subsection{Лирическое отступление}
Есть три основных источника спектральных последовательностей 
\begin{itemize}
    \item Спектральная последовательность фильтрованного комплекса
    \item Спектральная последовательность двойного комплекса
    \item Спектральная последовательность точной пары (Хатчер, Алгебраическая топология)
\end{itemize}
\subsection{Двойной комплекс}
\begin{to_def}
$K^{pq}$ -- двойной комплекс\footnote{$p$-- горизонтально, $q$ -- вертикально}
\[d^{\rightarrow}\colon K^{pq}\to K^{p+1, q}\]
\[d^{\uparrow}\colon K^{pq}\to K^{p, q+1}\]
\[d^{\rightarrow}d^{\uparrow} + d^{\uparrow}d^{\rightarrow} = 0\]
\[d^{\rightarrow}^2 = d^{\uparrow}^2 = 0\]
\end{to_def}
\begin{to_def}
Тотальный комплекс
\[Tot^{\oplus}(K^{\bullet\bullet})^n = \underset{p+q=n}{\oplus}K^{pq}\]
\[Tot^{\prod}(K^{\bullet\bullet})^n = \underset{p+q=r}{\prod}K^{pq}\]
\end{to_def}
\begin{to_claim}
Если комплекс находится в первом квадранте, то 
\[Tot^{\oplus} \cong Tot^{\prod}\]
\end{to_claim}
\begin{to_claim}
Пусть $K^{\bullet\bullet}$ -- ограничен (лежит в первом квадранте), а его строки \underline{или} столбцы ацикличны. Тогда $Tot(K)^\bullet$ -- ацикличен.
\end{to_claim}
\[(d^v + d^h)^2 = d^v^2 + d^h^2 + d^v d^h + d^h d^v = 0 \]
\[
d(z_0, z_1, z_2, \ldots z_n) = 0
\]
\begin{align*}
    d^h z_0 + d^v z_1 = 0\\
    d^h z_1 + d^v z_2 = 0\\
    \ldots\\
    d^v z_n + dz_{n-1} = 0
\end{align*}
\begin{align*}
    b_0 \ldots b_{n+1}\\
    b_0 = 0\\
    \exists b_1\colon
\end{align*}
\bee
\begin{tikzcd}
    & \arrow[d]&\arrow[d] &\arrow[d] & \\
    0&\arrow[l]K^{01}\arrow[d]& K^{11}\arrow[l]\arrow[d]&K^{21}\arrow[l]\arrow[d]&\arrow[l]\\
    0&\arrow[l]K^{00}\arrow[d]& K^{10}\arrow[l]\arrow[d]&K^{20}\arrow[l]\arrow[d]&\arrow[l]\\
     & 0 & 0 & 0 &
\end{tikzcd}
\eee
\end{document}