\documentclass[../main.tex]{subfiles}
\begin{document}
\section{Локализация, условия Оре}
Рассмотрим категорию $\mathcal{C}$ и $S\in \Hom_\mathcal{C}$ --- произвольное множество морфизмов, замкнутое относительно композиции. 
Предположим, мы бы хотели рассмотреть похожую на неё категорию с тем различием, что все морфизмы семейства $S$ в ней обратимы. 
Сконструировать её не очень сложно: мы навино добавляем обратные морфизмы и все композиции с ними. 
Естественно, морфизмов в итоге станет больше и появится естественное вложение $F$. 
Тем не менее, такая конструкция не очень тривиальна: несмотря на простоту построения, мы не знаем как она работает. 
Более того, категорий, удовлетворяющих нашему условию, может быть много. 
Оказывается, что именно самая наивная конструкция обладает следующим универсальным свойством в 2-категории всех категорий, которым мы такое построение и определим.

\begin{to_suj}
  Существует единственная с точностью до эквивалентности категория $\mathcal{C}[S^{-1}]$ с функтором $F:\mathcal{C}\to \mathcal{C}[S^{-1}]$, обладающая универсальным свойством: для любого $T:\mathcal{C}\to \mathcal{D}$, переводящего $S$ в изоморфизмы, существует единственный функтор $G:\mathcal{C}[S^{-1}] \to \mathcal{D}$, делающий следующую диаграмму коммутативной.

  \begin{equation}\label{loc_univ}
  \xymatrix{
      \mathcal{C}\ar[rd]^T \ar[r]^F & \mathcal{C}[S^{-1}]\ar@{-->}[d]^G\\
  & \mathcal{D} }
\end{equation}

\end{to_suj}

\begin{proof}
  Для доказательства приведем явную конструкцию категории $\mathcal{C}[S^{-1}]$. 
  Объекты в ней будут точно такие же: $\text{Ob}\mathcal{C}[S^{-1}] = \text{Ob}\mathcal{C}$. 
  Обратные морфизмы введём следующим образом:

  Назовём \emph{словом} формальный набор букв, обозначающий морфизмы из $\mathcal{C}$. 
  Добавим к буквам обратные $s^{-1}$ морфизмы для семейства $S$. 
  Слова должны состоять из компонуемых букв, т. е. в слове $fg$ конец морфизма $f$ и начало морфизма $g$ должны совпадать. 
  Будем считать, что начало и конец морфизма $s^{-1}$ -- это соответственно конец и начало $s$. 
  Введём на словах отношение эквивалентности, удовлетворяющее следующим правилам:
  \begin{itemize}
    \item $(f)(g) \sim (fg)$
    \item $(s)(s^{-1}) \sim \text{Id}_{A}$, где $A$ --- объект, являющийся началом $s^{-1}$.
    \item $(f)(s)(s^{-1})(g) \sim (f)(g)$.
  \end{itemize}
  Классы эквивалентности положим морфизмами в $\mathcal{C}[S^{-1}]$. 
  Легко проверить, что они удовлетворяют определению категории и каждый морфизм в $\mathcal{C}[S^{-1}]$ обратим.

  Тогда функтор $F$ определим как тождественно действующий на объектах и как вложение на морфизмах. 
  $G$ определим как действующий аналогично $T$ на объектах и морфизмах, являющихся образами $F$. 
  Появившиеся обратные морфизмы $G$ будет отправлять в соответствующие обратные в $\mathcal{D}$.
\end{proof}

\begin{to_com}
  Конструкция, приведённая в доказательстве, тесно связана с понятием локализации кольца $R$ мультипликативно замкнутым подмножеством $S$. 
  Мы также чисто формально добавляем обратный элемент каждому из заданного подмножества, рассматривая дроби вида $m/s$, $m\in R, s\in S$. 
  Тогда нашим отношением эквивалентности станет не что иное, как сокращение дробей: $r_1/s_1\sim r_2/s_2 \iff \exists t\in S:t(s_1r_2 - s_2r_1)=0$. 
  Причем такая конструкция имеет точно такое же универсальное свойство в категории колец!
\end{to_com}

Доказанное утверждение гарантирует корректность следующего определения.
\begin{to_def}
  Пусть $\mathcal{C}$ --- локально малая категория, $S\in \text{Hom}_\mathcal{C}$ --- произвольное множество морфизмов, замкнутое относительно композиции. 
  Локализацией $\mathcal{C}[S^{-1}]$ категории $\mathcal{C}$ по локализующему семейству $S$ будем называть категорию с универсальным свойством \eqref{loc_univ}.
\end{to_def}

  С помощью данного объекта введём пока что бесполезное определение.
  \begin{to_def}\label{DerCat}
    Производной категорией $\mathcal{D}(\mathcal{A})$ категории $\mathcal{A}$ будем называть локализацию гомотопической категории по квазиизоморфизмам.
    \begin{equation*}
      \mathcal{D}(\mathcal{A}) \cong \mathcal{K}(\mathcal{A})[Qis^{-1}]
    \end{equation*}
\end{to_def}

Введённое определение, конечно, не дают никакого понятия того, как такая категория устроена. 
Однако это устройство становится проще, если на семейство наложить \emph{условия Оре}.

\begin{to_def}[Условия Оре]
    Назовём семейство морфизмов $S$ в категории $\mathcal{C}$ локализующим, если оно удовлетворяет парам условий (указанным ниже и двойственным им).
    \begin{itemize}
      \item Все тождественные морфизмы лежат в $S$ и $S$ замкнуто относительно композиции.
      \item Для любого морфизма $f:X\to Z$ и морфизма $s:Y\to Z$ из семейства $S$ найдется объект $W$ и два морфизма $g\in \Hom_\mathcal{C}(W, Y)$, $t\in \Hom_\mathcal{C}(W, X), t\in S$, делающие следующую диаграмму коммутативной.
	\begin{equation}\label{ore_2_cond}
	  \xymatrix{W \ar[r]^g \ar[d]_t & Y\ar[d]^s\\
	  X \ar[r]_f & Z}
	\end{equation}
      \item Если известно, что для $s\in S$ $sf_1=sf_2$, то найдется $t\in S$ такой, что $f_1t=f_2t$. 
	Или иначе, если у двух морфизмов нашёлся левый уравнитель, то найдется и правый.
    \end{itemize}
      \label{ore_conditions}
    \end{to_def}

    При таких условиях оказывается, что можно дать более явное описание локализации в терминах \emph{домиков}.

     \begin{to_def}
      1. Назовём (левым) домиком $(s, f)$ тройку объектов $X, Y, Z$ и пару морфизмов $f\in \text{Hom}_\mathcal{C}(Z, Y)$, $s \in \text{Hom}_\mathcal{C}(Z, X)$, $s\in S$.
      \begin{equation}\label{sample_home}
	\xymatrix{&Z \ar[ld]_s \ar[rd]^f&\\
	X&&Y}
      \end{equation}

      2. Назовём два (левых) домика эквивалентными, если они ``достраиваются до большого домика''. $(s, f)\sim (t, g)$, если существует объект $Z'''$ и морфизмы $v, h$, $v\in S$, делающие следующую диаграмму коммутативной.
      \begin{equation}\label{home_equivalence}
	\xymatrix{&&Z'''\ar[ld]_v\ar[rd]^h&&\\
	  &Z'\ar[ld]_s\ar[rrrd]^f&&Z''\ar[llld]_t\ar[rd]^g&\\
	X&&&&Y}
      \end{equation}
      Т. е. существует ``больший'' домик $(sv, gh)$.
      \label{homescapes}

      3. Композиций двух (левых) домиков $(v, g)$, $(s, f)$ называется домик $(st, gl)$, делающий следующую диаграмму коммутативной.
	\begin{equation}\label{home_composition}
	\xymatrix{&&Z'\ar[ld]_t\ar[rd]^l&&\\
	  &X'\ar[ld]_s\ar[rd]^f&&Y'\ar[ld]_v\ar[rd]^g&\\
	X&&Y&&Z}
      \end{equation}

    \end{to_def}

    Следующее предложение помимо доказательства корректности определения выше показывает, что классы эквивалентности домиков удовлетворяют определению морфизмов в категории.
    \begin{to_suj}

      \begin{itemize}
	\item[1] Отношение, заданное \eqref{home_equivalence}, является отношением эквивалентности.
	\item[2] Композиция \eqref{home_composition} двух любых домиков определена. Более того, она не зависит от представителя класса эквивалентности.
	\item[3] Композиция \eqref{home_composition} домиков ассоциотивна на классах эквивалентности. Также существует единичный домик $(id, id)$
      \end{itemize}
      \label{home_correctness}
    \end{to_suj}
    \begin{proof}
      1. Симметричность и рефлексивность очевидны. 
      Проверим транзитивность. 
      Рассмотрим три домика $(f_A^i, f_B^i)$, $f_A^i\in \text{Hom}_\mathcal{C}(X^i, A)$,  $f_B^i\in \text{Hom}_\mathcal{C}(X^i, B)$, с вершинами $X^i$, $i=1, 2, 3$. 
      Пусть эквивалентны домики 1, 2 и 2,3. 
      Докажем эквивалентность 1, 3. 
      Используя второе условие Оре существует объект $Z$ с морфизмами $f_1, f_2$, причем $f_1$ из локализующего семейства.
      \begin{equation*}
	\xymatrix{W\ar[r]^w\ar[d]_{w_1}\ar[rrd]_{w_2}&Z\ar[ld]_{f_1}\ar[rd]^{f_2}\\
	  Z^1\ar[d]_{\alpha_1}\ar[rd]^{\alpha_2}&&Z^2\ar[d]^{\beta_2}\ar[ld]_{\beta_1}\\
	  X^1\ar[d]\ar[rrd]&X^2\ar[ld]\ar[rd]&X^3\ar[d]\ar[lld]\\
	A&&B}\Leftarrow \text{2 условие Оре \eqref{ore_2_cond}}
	\xymatrix{Z\ar[r]^{f_2}\ar[d]_{f_1}&Z^2\ar[d]^{\beta_1}\\
	Z^1\ar[r]^{\alpha_2}&X^2}
      \end{equation*}
Отметим, что диаграмма выше не является коммутативной. 
Для того, чтобы добиться коммутативности воспользуемся 3 условием Оре. 
Мы знаем, что верхний квадрат коммутирует: $f_A^2\alpha_2f_1 = f_A^2\beta_1f_2$. 
Тогда существует объект $W$ с морфизмом из локализующего семейства $w:W\to Z: \alpha_2f_1w=\beta_1f_2w$. 
Теперь, взяв $w_1 = f_1w$, $w_2 = f_2w$, получим уже коммутативную диаграмму с объектом $W$ вместо $Z$. 
Тогда искомым домиком  будет объект $W$ с морфизмами $\alpha_1w_1$ и $\beta_2w_2$.\\
	2.Пусть домики с вершинами $Z_1, Z_2$ эквивалентны и берутся в композиции с домиком с вершиной $M$.
    \begin{equation*}
      \xymatrix{Z\ar[d]\ar[rd]&X_1\ar[ld]\ar[rd]&X_2\ar[ld]\ar[d]\\
	Z_1\ar[d]\ar[rd]&Z_2\ar[ld]\ar[d]&M\ar[d]\ar[ld]\\
      A&B&C}
    \end{equation*}
Применим второе условие Оре \eqref{ore_2_cond} два раза и, соответственно, получим искомый домик.
    \begin{equation*}
      \xymatrix{&&F\ar[rd]\ar[ld]\ar[dd]\ar@/^3pc/[rrdd]\\
	&F_1\ar[ld]\ar[rd]&&F_2\ar[rd]\ar[llld]\\
      A&&B&&C}
    \end{equation*}
3. Покажем, что $(f_1, g_1)( (f_2, g_2)(f_3, g_3) ) \sim ( (f_1, g_1)(f_2, g_2) )(f_3, g_3)$.
  \begin{equation*}
    \xymatrix{&&Y_1\ar[ld]\ar[rd]&&Y_2\ar[ld]\ar[rd]\\
      &X_1\ar[rd]\ar[ld]&&X_2\ar[rd]\ar[ld]&&X_3\ar[ld]\ar[rd]\\
    A&&B&&C&&D}
  \end{equation*}
  $Y_1, Y_2$ --- вершины композиций первого и второго, второго и третего домиков соответственно. Снова применим 2 условие Оре \eqref{ore_2_cond} и получим для итоговых композиций $Z_1, Z_2$ эквивалентность:
  \begin{equation*}
    \xymatrix{&&Z\ar[rd]\ar[ld]\\
      &Z_1\ar[dl]\ar[d]\ar[rrrd]&&Z_2\ar[d]\ar[llld]\\
    X_1&Y_1\ar[l]&&Y_2\ar[r]&X_3}
  \end{equation*}
    \end{proof}
Теперь займёмся явным построением локализации категории семейством, удовлетворяющим условиям Оре.  Искомый морфизм $F:\mathcal{A} \to \mathcal{A}[S^{-1}]$ будет действовать тождественно на объектах.  На морфизмах положим:
       \begin{equation}\label{ore_functor}
	 f\in \text{Hom}_\mathcal{C}(X,Y) \to \xymatrix{&X\ar[ld]_{id}\ar[rd]^{f}\\X&&Y}
      \end{equation}
Под домиком, естественно, понимается его класс эквивалентности.  То, что такая конструкция даёт локализацию и удовлетворяет аксиомам функториальности, конечно, под вопросом.  Для начала можно проверить некоторые факты. Проверим, например, что образ локализующего семейства состоит из обратимых морфизмов. Действительно, обратным к домику $(id, s)$ будет домик $(s, id)$, так как следующая диаграмма коммутирует:
\begin{equation*}
	\xymatrix{&&X\ar[rd]^{id}\ar[ld]_{id}\\
	  &X\ar[rd]^s\ar[ld]_{id}&&X\ar[rd]^{id}\ar[ld]_s\\
	X&&Y&&Z}
	\end{equation*}
Верность аксиомы функториальности $F(fg) \sim F(f)F(g)$ следует из коммутативности следующей диаграммы:
      \begin{equation*}
	\xymatrix{&&X\ar[ld]_{id}\ar[rd]^f\\
	  &X\ar[ld]_{id}\ar[rd]^f&&Y\ar[rd]^g\ar[ld]_{id}\\
	X&&Y&&Z}
\end{equation*}
Очевидно, домик выше эквивалентен $F(fg)$. Наконец, проверим, что домики удовлетворяют универслаьному свойству и докажем следуещюю теорему.
\begin{to_thr}
 Пусть $S$ --- локализующее семейство локально малой категории $\mathcal{C}$. Тогда $\Hom_{\mathcal{C}[S^{-1}]}(X, Y)$ --- в точности классы эквивалентности домиков вида \eqref{sample_home}.
 \label{ore_loc_desc}
 \end{to_thr}
 \begin{proof}
 Пусть $T:\mathcal{C}\to \mathcal{D}$ переводит семейство $S$ в изоморфизмы. Построим функтор $G$, делающий диаграмму \ref{loc_univ} коммутативной.
На объектах $G$ должен определяться также, как и $T$. Пусть $\varphi$ --- морфизм в $\mathcal{С}[S^{-1}]$, представляющийся домиком $(s, g)$.  Возьмем его композицию $\varphi\cdot (1, s) = (1, f)$, где из построения \eqref{ore_functor} $(1, s) = F(s)$, $(1, f) = F(f)$. Т. е. мы получили, что
\begin{equation*}
	\varphi\cdot F(s) = F(f)\quad {}^{G(\cdot)}
	 \Rightarrow G(\varphi)\cdot T(s) = T(f)\quad
	 \Rightarrow G(\varphi) = T(f)T(s)^{-1},
      \end{equation*}
где последнее равенство даёт явное построение $G$ и, следовательно, доказывает универсальность.
\end{proof}
\end{document}