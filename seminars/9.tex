\documentclass[../main.tex]{subfiles}
\begin{document}
\section{Функтор $Ext$ по Йонеде}
\begin{to_def}
Пусть $\A$ -- абелева категория. Расширением объекта $C$ с помощью объекта $A$ длины $1$ будем называть короткую точную последовательность вида:
\bee
\begin{tikzcd}[sep=small]
    0 \arrow[r]& A \arrow[r]& B\arrow[r]& C\arrow[r]& 0
\end{tikzcd}
\eee
Аналогично расширение длины $n$ определим как:
\bee
\begin{tikzcd}[sep=small]
    0 \arrow[r]& A \arrow[r]& B_1\arrow[r]&\ldots \arrow[r]&B_n\arrow[r]& C\arrow[r]& 0
\end{tikzcd}
\eee
\end{to_def}
\begin{to_not_def}
Два расширения называются эквивалентными, если существует морфизм расширений как комплексов $\alpha = \{\alpha_i\}_i^{n}$
\bee
\begin{tikzcd}
        0 \arrow[r]& A\arrow[d, "id"] \arrow[r]& B_1\arrow[d, "\alpha_1"]\arrow[r]&\ldots \arrow[r]&B_n\arrow[d, "\alpha_n"]\arrow[r]& C\arrow[d, "id"]\arrow[r]& 0\\
        0 \arrow[r]& A \arrow[r]& B'_1\arrow[r]&\ldots \arrow[r]&B'_n\arrow[r]& C\arrow[r]& 0\\
\end{tikzcd}
\eee
\end{to_not_def}
\begin{to_com}
Введённое отношение эквивалентности не является отношением эквивалентности. Правильное определение должно являться минимальным отношением такого вида. Правильнее было бы сказать, что расширения должны быть квазиизоморфны как комплексы, однако, проверка квазиизоморфности комплексов крайне алгоритмически сложна. Но в силу того, что мы рассматриваем ацикличные комплексы, проверка квазиизоморфности может быть выполнена за $2n$ шагов.
\end{to_com}
В случае расширений длины $1$ по лемме о 5 два расширения будут эквивалентны тогда и только тогда, когда изоморфны средние члены последовательностей. 
\bee
\begin{tikzcd}
        0 \arrow[r]& A\arrow[d, "id"] \arrow[r]& B\arrow[d, "\cong"]\arrow[r]& C\arrow[d, "id"]\arrow[r]& 0\\
        0 \arrow[r]& A \arrow[r]& B'\arrow[r]&C\arrow[r]& 0
\end{tikzcd}
\eee
Таким образом может быть определено множество классов эквивалентности расширений $Ext^1(C, A)$.
\begin{to_ex}
Неэквивалентные расширения $Ext^1(\Z_2, \Z)$
\bee
\begin{tikzcd}
        0 \arrow[r]& \Z\arrow[d, "id"] \arrow[r, "\cdot 2"]& \Z\arrow[r]& \Z_2\arrow[d, "id"]\arrow[r]& 0\\
        0 \arrow[r]& \Z \arrow[r]& \Z\oplus\Z_2 \arrow[r]&\Z_2\arrow[r]& 0
\end{tikzcd}
\eee
\end{to_ex}
\begin{to_ex}
$Ext^1(\Z_m, A) = A /mA$
\bee
\begin{tikzcd}[sep=small]
        0 \arrow[r] & \underset{g}{A} \arrow[r, "\alpha"] & \underset{\alpha g = m u}{B}\arrow[r, "\beta"]& \Z_n(c) \arrow[r]& 0\\
                    &                       &a\arrow[r, mapsto]&     c             &  \end{tikzcd}
\eee

\begin{align*}
    &\forall b \in B,\quad h \in \{1, ..., m-1\} \\
    \alpha b &= a + h u \\
    mb &= m\alpha a + h\underset{\in\ker \beta }{(mu)}\\
    mu &= \alpha g, \quad g \in A
\end{align*}
\begin{align*}
    (\alpha a + h u )+(\alpha a' + h'u) = \begin{cases}\alpha(a + a') + (h + h')u, & h + h' \le m \\ \alpha(a+a'+g) + (h + h' - m)u, & h+h' \geq m \end{cases}
\end{align*}
Задание элемента $g$ однозначно задаёт сложение в группе $B$. Несмотря на то, что сам элемент $g$ определён неоднозначно, класс смежности в образе определён однозначно. Это и означает, что существует биекция между классами смежности и всевозможными расширениями $\Z_m$. 
\end{to_ex} 
\subsection{Сложение по Бэру}
Приведённый пример наводит на мысль, что на множестве $Ext$' ов может быть задана групповая структура. Может и будет задана. Для этого введём сложение расширений по Бэру:
\begin{to_claim}
Пусть есть расширение \begin{tikzcd} E \colon A \arrow[r, "\alpha"] & B \arrow[r, "\beta"] & C \end{tikzcd} и морфизм $\gamma\colon C' \rightarrow C$. Тогда $\exists ! $ расширение $E'$, которое начинается на $A$ и заканчивается на $C'$, задаваемое морфизмом комплексов $(id_{A}, \delta, \gamma): E' \rightarrow E$.
\bee
\begin{tikzcd}
        0 \arrow[r]& \overset{a}{A}\arrow[d, "id_{A}"] \arrow[r]& \overset{(a, 0)}{B\underset{C}{\times}C'} \arrow[d, "\delta"]\arrow[r, "\beta'"]& C'\arrow[d, "\gamma"]\arrow[r]& 0\\
        0 \arrow[r]& A \arrow[r, "\alpha"]& \underset{a}{B}\arrow[r, "\beta"]&\underset{0}{C}\arrow[r]& 0
\end{tikzcd}
\eee
\begin{align*}
    B \underset{C}{\times} C' &= \{ (b, c)\in B\oplus C' |\quad \beta\delta (b) = \gamma\beta' (c)\}\\
    \delta(b, c) &= b\\
    \beta'(b, c') &= c'
\end{align*}
Проведя двойственные рассуждения можем получить аналогичное утверждение для $\gamma'\colon A \rightarrow A'$\\
\bee
\begin{tikzcd}
        0 \arrow[r]& A\arrow[d, "\gamma'"] \arrow[r, "\alpha"]& B \arrow[d, "\delta"]\arrow[r, "id_C"]& C\arrow[d]\arrow[r]& 0\\
        0 \arrow[r]& A' \arrow[r, "\alpha'"]& A'\underset{A}{\coprod} B \arrow[r]&C\arrow[r]& 0
\end{tikzcd}
\eee
\end{to_claim}

Опишем теперь алгоритм сложения двух расширений:
\bee
\begin{tikzcd}[sep=small]
        0 \arrow[r]& A \arrow[r]& B\arrow[r]& C\arrow[r]& 0\\
        0 \arrow[r]& A \arrow[r]& B'\arrow[r]&C\arrow[r]& 0 
\end{tikzcd}
=
\begin{tikzcd}
    0 \arrow[r]& A \arrow[r]& B'''\arrow[r]&C\arrow[r]& 0 
\end{tikzcd}
\eee
\newline
\bee
\begin{tikzcd}
        0 \arrow[r]& A \arrow[r]&A \underset{A\oplus A}{\coprod} B''\arrow[r] = B'''& C \arrow[r]&0\\
        0 \arrow[r]& A\oplus A\arrow[d] \arrow[u]\arrow[r]& (B\oplus B')\underset{C\oplus C}{\times}C \arrow[u]\arrow[d]\arrow[r] = B''& C\arrow[u]\arrow[d, "\Delta"]\arrow[r]& 0\\
        0 \arrow[r]& A\oplus A \arrow[r]& B\oplus B'\arrow[r]&C\oplus C\arrow[r]& 0
\end{tikzcd}
\eee
Итак, мы ввели сложение расширений по Бэру, задав на них групповую структуру. Несложно убедиться, что нейтральным по сложению элементом данной группы является тривиальное расширение \begin{tikzcd}[sep = small] 0\arrow[r]& A \arrow[r]& A\oplus C \arrow[r]& C \arrow[r]& 0\end{tikzcd}.
\subsection{Сложение $Ext$ длины $n$}
Пусть имеем два расширения $\xi, \xi'$ и морфизм комплексов, тождественный на $A$ и $B$.
\bee
\begin{tikzcd}[sep=small]
\xi\colon& 0\arrow[r]& B\arrow[r]\arrow[d, "id_B"]& X_n\arrow[r]&\cdots\arrow[r]& X_1\arrow[r]& A\arrow[d, "id_A"]\arrow[r]&0\\
\xi'\colon& 0\arrow[r]& B\arrow[r]& X'_n\arrow[r]&\cdots\arrow[r]& X'_1\arrow[r]& A\arrow[r]&0
\end{tikzcd}
\eee
Тогда суммой по Бэру двух таких расширений будет комплекс
\bee
\begin{tikzcd}[sep=small]
0\arrow[r]& B\arrow[r]& X''_n\arrow[r]& X_{n-1}\oplus X'_{n-1}\arrow[r]&\cdots\arrow[r]& X_2\oplus X'_2\arrow[r]& X''_1\arrow[r]& A\arrow[r] & 0\\
          &           &   \underset{B}{X_n \coprod X_n'}                &               &                        & &X_1\underset{A}{\times}X_1'&           
\end{tikzcd}
\eee
Пусть есть расширение $Ext^1(C, A)$ и морфизм $\alpha\colon A\rightarrow X$. Если $\alpha$ продолжаем на $B$, то есть $\exists \alpha' \colon B \rightarrow X$, тогда можно построить диаграмму:
\bee
\begin{tikzcd}[sep=small]
    0\arrow[r]& A \arrow[r]\arrow[d, "\alpha"] & B \arrow[r]\arrow[d, "(\alpha' \pi)"]& C \arrow[r]\arrow[d, "id"]& 0\\
    0\arrow[r]& X \arrow[r] & X \oplus C \arrow[r]& C \arrow[r]& 0\\
              &             & X \underset{A}{\coprod} B
\end{tikzcd}
\eee
Таким образом можно сказать, что существование нетривиальных $Ext$' ов препятствует продолжению морфизмов.
\subsection{Умножение $Ext$' ов}
Пусть имеем 2 расширения длины 1 $\gamma \in Ext^1(C, A)$ и $\delta \in Ext^1(Z, C)$. Опустив $C$ в цепочке морфизмов мы получим некоторый элемент $Ext^2(Z, A)$.
\bee
\begin{tikzcd}[sep=small]
    \gamma \colon & 0 \arrow[r] & A \arrow[r] & X \arrow[r] & C \arrow[r]&0\\
    \delta \colon & 0 \arrow[r] & C \arrow[r] & Y \arrow[r] & Z \arrow[r]&0\\
    \gamma * \delta \colon & A \arrow[r] & X \arrow[r] & \cancel{C} \arrow[r]& Y \arrow[r] & Z\arrow[r]&0
\end{tikzcd}
\eee
\[
* \colon Ext^m (C, A) \times Ext^n(Z, C) \rightarrow Ext^{n+m}(Z, A)
\]
Таким образом может быть задана структура градуированной алгебры\footnote{градуированной некоммутативной $\Z_{\ge 0}$--алгебры при совпадающих аргументах, если не фиксировать аргумент, то алгебра градуирована целыми неотрицательными числами и парами объектов
} $Ext_{R}^\bullet (A, A)$, $A\in R-mod $.\\
$Ext_{R}^\bullet(A, B)$ -- это модуль над $Ext_{R}^\bullet(A, A)$, $A, B \in R\text{-mod} $
\begin{to_claim}\label{pred}
\[Ext_{\A}^n(A, B) \cong Hom_{\D{A}}(A[0], B[n])\]
\end{to_claim}
\begin{proof}
$(\Rightarrow)$ Каждому расширению длины $n$ можно сопоставить домик в производной категории вида: 
\bee
    \begin{tikzcd}
        0\arrow[r]& B\arrow[r]\arrow[ld]& X_n\arrow[r]&\cdots\arrow[r]& X_1\arrow[r]\arrow[rd]& A\\
        B[-n]     &                     &             &               &                       &A[0]
    \end{tikzcd}
\eee
$(\Leftarrow)$ Каждый домик из $Hom_{\D{A}}(A[0], B[n])$ представляет класс квазиизоморфных комплексов с нетривиальными когомологиями только в нулевом члене. Обрезав такой комплекс, мы получим расширение длины $n$.
\bee
\begin{tikzcd}[sep=small]
       & K^\bullet \arrow[ld, "qis"'] \arrow[rd] &        \\
{A[0]} &                                 & {B[n]}
\end{tikzcd}
\eee
\bee
\begin{tikzcd}[sep=small]
K^{\bullet} \arrow[r, "\tau_{ \curlyeqsucc n}", Rightarrow] & B \arrow[r] & \cancel{0} \arrow[r] & Z^n(K^{\bullet})=B^n(K^{\bullet}) \arrow[r] & K^{n+1} \arrow[r] & \ldots \arrow[r] & K^0 \arrow[r] & A
\end{tikzcd}
\eee
\end{proof}
\begin{to_claim}
\[
Ext^n(A, -) = R^n Hom(A^\bullet, -)
\]
\end{to_claim}
\begin{proof}
Возьмём объект $B$ и его инъективную резольвенту $\mathcal{I}_B^\bullet$. С одной стороны из \ref{pred}:
\[
Ext^i_{\A}(A, B) = Hom_{\D{A}} (A[0],\mathcal{I}_B^\bullet [i]) = Hom_{\K{A}}(A[0],\mathcal{I}_B^\bullet [i])
\]
С другой стороны из определения классического производного функтора и используя равенство $\hHom^\bullet(A[0], \mathcal{I}_B^\bullet) \cong Hom(A, B)$, получим:
\[
R^i Hom(A, B) = H^i(\hHom^\bullet(A[0], \mathcal{I}_B^\bullet))=Hom_{\K{A}}(A[0], \mathcal{I}_B^\bullet [i])
\]
\end{proof}
\begin{to_claim}
Не бывает отрицательных $Ext'$ов.\\
$\forall$ $A$, $B$ $\in$ $\mathcal{A}$ $Ext^n (A, B) = 0$, $n<0$.
\begin{proof}
Пусть $0 \neq f \in \Hom_{\mathcal{D(A)}}(A, B[-n])$
\bee
\begin{tikzcd}[sep=small]
    & M \arrow[ld]\arrow[rd]& \\
  A & & B[-n]  
\end{tikzcd}
\sim\text{ }
\begin{tikzcd}[sep=small]
    & M \arrow[ld, "\tau_{n \curlyeqsucc 0}"']\arrow[rd]& \\
  A & & B[-n]  
\end{tikzcd}
\eee
\end{proof}
\end{to_claim}
\end{document}