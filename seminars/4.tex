\section{Проективные резольвенты в гомотопической категории, абелевость гомотопической категории}
Мы движемся к тому, чтобы определить производную категорию -- это наша стратегическая цель. Наша тактическая цель на данном этапе -- установить  нужные нам свойства фунткора взятия проективной резольвенты.
\begin{to_com}
Проективная резольвента с точностью до гомотопической эквивалентности определена однозначно.
\end{to_com}
\begin{to_suj}
Пусть $M, N \in \textbf{Ob}(\A)$. $P_\bullet$ -- проективная резольвента $M$, $Q_\bullet$ -- какая-то резольвента $N$.
Тогда 
\begin{align*}
\Hom_{\mathcal{K}(A)}(P_\bullet, Q_\bullet) = \Hom_{A}(M, N)
\end{align*}
\end{to_suj}
\begin{proof}
Рассмотрим морфизм комплексов $f\colon Q_\bullet \to N[0] $ и его конус \begin{tikzcd}[sep=small] Q_\bullet  \arrow[r] & N[0] \arrow[r] & C(f) \arrow[r] & Q_\bullet [1] \end{tikzcd}\\
\bee
\begin{tikzcd}[sep=small]
\ldots \arrow[r] & Q_1 \arrow[r] \ar[d] & Q_0 \arrow[r]  \arrow[d] & 0 \arrow[r] \arrow[d] & ... \\
\ldots \arrow[r] & 0 \arrow[r] & N \arrow[r] & 0 \arrow[r] & ...
\end{tikzcd}
\eee
$f$ -- $qis$ $\Leftrightarrow$ $C(f)$-- ацикличен $\Rightarrow$ $\forall g: P_\bullet \rightarrow C(f) \hookrightarrow g \sim 0$. Применим точный функтор $\Hom(P_\bullet, -)$ к короткой точной последовательности 
\begin{center}
\begin{tikzcd}[sep=small]
\underset{0}{\Hom(P_\bullet, N[-1])} \ar[r] &\Hom(P_\bullet, Q_\bullet) \ar[r, "\cong"] &\Hom(P_\bullet, N[0]) \ar[r] &\underset{0}{\Hom(P_\bullet, C(f))} \ar[r] & \Hom(P_\bullet, N[1]) \ar[r] & ...\\
\end{tikzcd}
\end{center}
В $\Hom(P_\bullet, N[0])$ нет нетривиальных гомотопий, а также $f d_1 = 0$ и $P_0/im d_1 = M$(универсальность коядра гомоморфизмов групп). Отсюда получаем:
\bee
\Hom_{\mathcal{K(A)}}(P_\bullet, N[0]) \cong \Hom(P_\bullet, N[0]) \cong \Hom(M, N)
\eee
\end{proof}
\begin{to_con}
Пусть $P^1_\bullet$, $P^2_\bullet$ -- проективные резольвенты $M$ $\Rightarrow$ $P^1_\bullet \sim P^2_\bullet$. 
\end{to_con}
\begin{to_con}
Проективная резольвента от модуля -- строгий полный функтор.
\bee
P_\bullet(-): \mathcal{A} \rightarrow \mathcal{K(A)}
\eee
\end{to_con}
\begin{to_com}
\begin{align*}
    \mathcal{A} \in \mathbf{Ab}&\Rightarrow Kom\mathcal{(A)}\\
    \mathcal{A} \in \mathbf{Ab}&\nRightarrow \mathcal{K(A)}
\end{align*}
\end{to_com}
\begin{to_ex}
$f: \Z \twoheadrightarrow \Z_p $ не является эпиморфизмом в $\mathcal{K(A)}$.
\begin{proof}
Предположим, что $f$ -- $epi$
\bee
\begin{tikzcd}[sep=small]
    \ldots \arrow[r] & 0 \arrow[r] \ar[d] & \Z \arrow[r]  \arrow[d, "f"] & 0 \arrow[r] \arrow[d] & \ldots = A\\
    \ldots \arrow[r] & 0 \arrow[r] & \Z_{p} \arrow[r] & 0 \arrow[r] & \ldots = C
\end{tikzcd}
\eee
Тогда существует разложение:
\bee
\begin{tikzcd} A \arrow[r, "f"] \arrow[rd, twoheadrightarrow] & C\\ & im f = B \arrow[u, hookrightarrow] \end{tikzcd}
\begin{tikzcd} A \arrow[r, "\alpha"]& B \arrow[r, "\beta"] & C\end{tikzcd}
\eee
\bee
\begin{tikzcd}[sep = small]
A\arrow[rr, bend left]\arrow[r]&B \arrow[r]&C(\alpha)\\
C(\beta)[-1]\arrow[rr, bend right]\arrow[r]&B \arrow[r]&C
\end{tikzcd}
\eee
\newpage
\bee
\begin{tikzcd}[sep=small]
    C(\beta)[-1]\arrow[r]&B \arrow[r]&C(\alpha)
\end{tikzcd}
\eee
\bee
\begin{tikzcd}
    \ldots\arrow[d] & \ldots\arrow[d] & \ldots\arrow[d]\\
    B_{-1} \arrow[r]\arrow[d] & B_{-1} \arrow[r]\arrow[d] & B_{-1}\oplus\Z\arrow[d]\\
    B_{0} \arrow[r]\arrow[d]\arrow[ru, "h_0" description ] & B_{0} \arrow[r]\arrow[d]\arrow[ru, "k_0" description ] & B_{0}\arrow[d]\\
    B_{1}\oplus\Z_p \arrow[r]\arrow[d]\arrow[ru, "h_1" description ] & B_{1} \arrow[r]\arrow[d]\arrow[ru, "k_1" description ] & B_{1}\arrow[d]\\
    \ldots & \ldots & \ldots
\end{tikzcd}
\eee
\begin{align*}
    s = \begin{cases}h^n, & n\neq 1\\ k^n, & n=1  \end{cases}\\
    d = dsd
\end{align*}
Таким образом получили, $B$ -- расщипим. 

\bee
\begin{tikzcd}
    \ldots\arrow[d] & \ldots\arrow[d] & \ldots\arrow[d]\\
    0 \arrow[r]\arrow[d] & 0 \arrow[r]\arrow[d] & B_{-1}\oplus\Z\arrow[d]\\
    B_{0} \arrow[r]\arrow[d]\arrow[ru, "h_0" description ] & B_{0} \arrow[r]\arrow[d]\arrow[ru, "k_0" description ] & B_{0}\arrow[d]\\
    B_{1}\oplus\Z_p \arrow[r]\arrow[d]\arrow[ru, "h_1" description ] & 0 \arrow[r]\arrow[d]\arrow[ru, "k_1" description ] & 0\arrow[d]\\
    \ldots & \ldots & \ldots
\end{tikzcd}
\eee
Но тогда имеем последовательность следующую морфизмов и противоречие:
\bee
\begin{tikzcd}
    B_0 \arrow[r]\arrow[rrrr, bend left, "id"] & \Z_p\arrow[rr, bend right, "0"'] \arrow[r] & B_0 \arrow[r] & \Z \arrow[r] & B_0
\end{tikzcd}
\eee
\end{proof}
\end{to_ex}
\begin{to_def}
$\mathcal{A}$ -- полупроста $\Leftrightarrow$ $\forall$ точная последовательность расщипима.
\end{to_def}
\begin{to_claim}
$\mathcal{K(A)}$ $\in$ $\mathbf{Ab}$ $\Leftrightarrow$ $\mathcal{A}$ -- полупроста(то есть $\forall$ точная последовательность расщипима).
\end{to_claim}
\begin{to_ex}
$\mathbf{Ab}$-- не полупроста. Пример нерасщипимой короткой точной последовательности:
\bee
\begin{tikzcd}[sep=small]
0 \arrow[r] &\Z \arrow[r, "\cdot p"] & \Z \arrow[r] & \Z_p \arrow[r] &  0
\end{tikzcd}
\eee
\end{to_ex}