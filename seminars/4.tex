\documentclass[../main.tex]{subfiles}
\begin{document}
\section{Проективные резольвенты в гомотопической категории, абелевость гомотопической категории}
Мы движемся к тому, чтобы определить производную категорию -- это наша стратегическая цель. Наша тактическая цель на данном этапе -- установить  нужные нам свойства фунткора взятия проективной резольвенты.
\begin{to_com}
Проективная резольвента с точностью до гомотопической эквивалентности определена однозначно.
\end{to_com}
\begin{to_suj}
Пусть $P_\bullet$ -- проективная резольвента $M$, $Q_\bullet$ -- какая-то резольвента $N$.
Тогда 
\begin{align*}
\Hom_{\mathcal{K}(A)}(P_\bullet, Q_\bullet) = \Hom_{A}(M, N)
\end{align*}
\end{to_suj}
\begin{proof}
Рассмотрим связывающий морфизм между резольвентой и объектом как морфизм комплексов $f\colon Q_\bullet \to N[0] $, тогда его конус -- это просто $P_\bullet\to Q \to 0$. Теперь, как обычно, имеем короткую точную последовательность, к которой можем применить функтор $\hHom(P_\bullet, - )$ и из леммы о зигзаге получим длинную точную последовательность:
\bee
\begin{tikzcd}[sep=small] Q_\bullet  \arrow[r] & N[0] \arrow[r] & C(f) \arrow[r] & Q_\bullet [1] \end{tikzcd}
\eee
%\bee
%\begin{tikzcd}[sep=small]
%\ldots \arrow[r] & Q_1 \arrow[r] \ar[d] & Q_0 \arrow[r]  \arrow[d] & 0 \arrow[r] \arrow[d] & ... \\
%\ldots \arrow[r] & 0 \arrow[r] & N \arrow[r] & 0 \arrow[r] & ...
%\end{tikzcd}
%\eee  
\bee
\begin{tikzcd}[sep=small]
\underset{0}{\Hom_{\K{A}}(P_\bullet, N[-1])} \ar[r] &\Hom_{\K{A}}(P_\bullet, Q_\bullet) \ar[r, "\cong"] &\Hom_{\K{A}}(P_\bullet, N[0]) \ar[r] &\underset{0}{\Hom_{\K{A}}(P_\bullet, C(f))} \ar[r] & \Hom_{\K{A}}(P_\bullet, N[1]) \ar[r] & ...
\end{tikzcd}
\eee
Так как $f$ -- квазиизоморфизм, то $C(f)$-- ацикличен. В силу \ref{PtoAc} $\forall g \in \Hom(P_\bullet, C(f)[i])$ выполнено, что $g \sim 0$. Также в $\Hom(P_\bullet, N[0])$ нет нетривиальных гомотопий, они все пропускаются через 0. А также $f d_1 = 0$ и $P_0/im d_1 = M$\footnote{универсальность коядра гомоморфизмов групп}. Отсюда получаем
\[
\Hom_{\K{A}}(P_\bullet, Q_\bullet) \cong \Hom_{\mathcal{K(A)}}(P_\bullet, N[0]) \cong \Hom(P_\bullet, N[0]) \cong \Hom(M, N)
\]
\end{proof}
\begin{to_con}
Пусть $P^1_\bullet$, $P^2_\bullet$ -- проективные резольвенты $M$ $\Rightarrow$ $P^1_\bullet \sim P^2_\bullet$. 
\end{to_con}
\begin{to_con}
Проективная резольвента от модуля -- строгий полный функтор.\ref{tactics}
    \begin{align*}
        \mathcal{P}\colon&\A \to\K{A}\\
         &A \mapsto P_\bullet(A)
    \end{align*}
\end{to_con}
\begin{to_com}
$\mathcal{P}$ не является фунткором в категоию комплексов, но является функтором в гомотопическую категорию.
\end{to_com}
\begin{to_com}
Понятно, что категория комплексов является абелевой, так как все ядра и коядра можно брат почленно, однако гомотопическая категория абелевой, вообще говоря, не является.
\begin{align*}
    \mathcal{A} \in \mathbf{Ab}&\Rightarrow Kom\mathcal{(A)}\\
    \mathcal{A} \in \mathbf{Ab}&\nRightarrow \mathcal{K(A)}
\end{align*}
\end{to_com}
\begin{to_ex}[Потеря эпиморфности при переходе в гомотопическую категорию]
$f: \Z \twoheadrightarrow \Z_p $ не является эпиморфизмом в $\mathcal{K(A)}$.
\begin{proof}
Предположим, что $f$ -- $epi$
\bee
\begin{tikzcd}[sep=small]
    \ldots \arrow[r] & 0 \arrow[r] \ar[d] & \Z \arrow[r]  \arrow[d, "f"] & 0 \arrow[r] \arrow[d] & \ldots = A\\
    \ldots \arrow[r] & 0 \arrow[r] & \Z_{p} \arrow[r] & 0 \arrow[r] & \ldots = C
\end{tikzcd}
\eee
В абелевой категории любой морфизм раскладывается в композицию $mono$ и $epi$.
\bee
\begin{tikzcd} A \arrow[r, "f"] \arrow[rd, twoheadrightarrow] & C\\ & im f = B \arrow[u, hookrightarrow] \end{tikzcd}
\begin{tikzcd} A \arrow[r, "\alpha"]& B \arrow[r, "\beta"] & C\end{tikzcd}
\eee
Теперь снова рассмотрим точные тройки из последовательности с конусами, на диаграмме ниже дуговые стрелки гомотопны 0
\bee
\begin{tikzcd}[sep = small]
A\arrow[rr, bend left]\arrow[r]&B \arrow[r]&C(\alpha)\\
C(\beta)[-1]\arrow[rr, bend right]\arrow[r]&B \arrow[r]&C
\end{tikzcd}
\eee
\bee
\begin{tikzcd}[sep=small]
    C(\beta)[-1]\arrow[r]&B \arrow[r]&C(\alpha)
\end{tikzcd}
\eee
Таким образом получили, что $B$ -- расщепим. Тогда, используя \ref{split1} мы можем представить его в виде
\bee
\begin{tikzcd}
    \ldots\arrow[d] & \ldots\arrow[d] & \ldots\arrow[d]\\
    B_{-1} \arrow[r]\arrow[d] & B_{-1} \arrow[r]\arrow[d] & B_{-1}\oplus\Z\arrow[d]\\
    B_{0} \arrow[r]\arrow[d]\arrow[ru, "h_0" description ] & B_{0} \arrow[r]\arrow[d]\arrow[ru, "k_0" description ] & B_{0}\arrow[d]\\
    B_{1}\oplus\Z_p \arrow[r]\arrow[d]\arrow[ru, "h_1" description ] & B_{1} \arrow[r]\arrow[d]\arrow[ru, "k_1" description ] & B_{1}\arrow[d]\\
    \ldots & \ldots & \ldots
\end{tikzcd}
\eee
\begin{align*}
    s = \begin{cases}h^n, & n\neq 1\\ k^n, & n=1  \end{cases}\\
    d = dsd
\end{align*}
И подобрать квазиизоморфный комплекс с когомологиями, сосредоточенными только в нулевом члене.
\bee
\begin{tikzcd}
    \ldots\arrow[d] & \ldots\arrow[d] & \ldots\arrow[d]\\
    0 \arrow[r]\arrow[d] & 0 \arrow[r]\arrow[d] & B_{-1}\oplus\Z\arrow[d]\\
    B_{0} \arrow[r]\arrow[d]\arrow[ru, "h_0" description ] & B_{0} \arrow[r]\arrow[d]\arrow[ru, "k_0" description ] & B_{0}\arrow[d]\\
    B_{1}\oplus\Z_p \arrow[r]\arrow[d]\arrow[ru, "h_1" description ] & 0 \arrow[r]\arrow[d]\arrow[ru, "k_1" description ] & 0\arrow[d]\\
    \ldots & \ldots & \ldots
\end{tikzcd}
\eee
Но тогда имеем следующую последовательность морфизмов и противоречие:
\bee
\begin{tikzcd}
    B_0 \arrow[r]\arrow[rrrr, bend left, "id"] & \Z_p\arrow[rr, bend right, "0"'] \arrow[r] & B_0 \arrow[r] & \Z \arrow[r] & B_0
\end{tikzcd}
\eee
\end{proof}
\end{to_ex}
\begin{to_def}
$\mathcal{A}$ -- полупроста $\Leftrightarrow$ $\forall$ точная последовательность расщепима.
\end{to_def}
\begin{to_claim}
$\mathcal{K(A)}$ -- абелева $\Leftrightarrow$ $\mathcal{A}$ -- полупроста.
\end{to_claim}
\begin{to_ex}[Нерасщепимая короткая точная последовательность]
\bee
\begin{tikzcd}[sep=small]
0 \arrow[r] &\Z \arrow[r, "\cdot p"] & \Z \arrow[r] & \Z_p \arrow[r] &  0
\end{tikzcd}
\eee
То есть $\mathbf{Ab}$ -- не полупроста.
\end{to_ex}
\end{document}