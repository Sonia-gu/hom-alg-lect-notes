\documentclass[../main.tex]{subfiles}
\begin{document}
\section{Конус}
\begin{to_def}
Сдвигом комплекса $K[n]$ назовём комплекс с объектами $K[n]^i = K^{n+i}$ и дифференциалами $d[n]^i = (-1)^nd^{i+n}$
\end{to_def}
\begin{to_def}
Конусом $C[f]^\bullet$ морфизма комплексов $f^\bullet:K^\bullet\to L^\bullet$ называется комплекс с объектами $C[f]^i = K^{i+1}\oplus L^i$ и дифференциалами $d^i = \begin{pmatrix}-d_K^{i+1}&0\\f&d_K^i\end{pmatrix}$.
\end{to_def}
Простая выкладка (индексы опущены для простоты) показывает корректность определения:
\begin{equation*}
d^2 = \begin{pmatrix}-d&0\\f&d\end{pmatrix}^2 = \begin{pmatrix}d^2&0\\-fd + df&d^2\end{pmatrix} = 0
\end{equation*}
\begin{to_suj}\label{3:cone_short_exact}
Для морфизма комплексов $f:K\to L$ следующая последовательность точна.
\begin{equation}\label{3:cone_short_exact_eq}
    \begin{tikzcd}
	0 & L & {C(f)} & {K[1]} & 0
	\arrow[from=1-1, to=1-2]
	\arrow["i", from=1-2, to=1-3]
	\arrow["\pi", from=1-3, to=1-4]
	\arrow[from=1-4, to=1-5]
\end{tikzcd}
\end{equation}
\end{to_suj}
\begin{proof}
Приведённая последовательность расщипима в каждом члене.
\end{proof}
\begin{to_thr}
$f$ --- квазиизоморфизм тогда и только тогда, когда $C(f)$ --- ацикличен.
\end{to_thr}
\begin{proof}
    Применим лемму о зигзаге \ref{B:zigzag_lemma} к последовательности \eqref{3:cone_short_exact_eq}.
\end{proof}
Дадим ещё несколько определений 
\begin{to_def}
%% продолжу с этого момента
$K^\bullet$ -- расщипимый комплекс, если 
\[\exists s^i\colon K^{i} \to K^{i-1} \quad \colon \quad dsd = d \]
\end{to_def}
\begin{to_def}
$K^\bullet$ -- стягиваемый, если $id_{K^\bullet}\sim 0$
\end{to_def}
\begin{to_ex}[простейший стягиваемый комплекс]\label{idM}
\be
\label{eq:vid2}
\begin{tikzcd}[sep=small] \ldots\arrow[r]& 0 \arrow[r]& M \arrow[r, "id"]& M \arrow[r]& 0 \arrow[r]& \ldots\end{tikzcd}
\ee
\end{to_ex}
\begin{to_ex}[простейший расщипимый комплекс]
\be
\label{eq:vid3}
\begin{tikzcd}[sep=small] \ldots\arrow[r]& 0 \arrow[r]& M \arrow[r]& 0 \arrow[r]& \ldots\end{tikzcd}
\ee
$\NB$ он не стягиваемый
\end{to_ex}
\begin{to_suj}
$(K^\bullet$ -- стягиваемый$)$ $\quad \Rightarrow \quad$ $(K^\bullet$ -- расщипимый$)$
\bee
\begin{tikzcd}
    \ldots\arrow[r]& K^{i-1}\arrow[r]\arrow[ld, "s^{i-1}"']\arrow[d, "id"']&K^{i}\arrow[r]\arrow[ld, "s^{i}"']\arrow[d, "id"']& K^{i+1}\arrow[r]\arrow[ld, "s^{i+1}"']\arrow[d, "id"']& \ldots\\
    \ldots\arrow[r]& K^{i-1}\arrow[r]&K^{i}\arrow[r]& K^{i+1}\arrow[r]& \ldots\\
\end{tikzcd}
\eee
\end{to_suj}
\begin{to_suj}
Любой стягиваемый комплекс является суммой сдвигов комплексов вида \ref{idM}. 
\end{to_suj}
\begin{to_suj}
$K^\bullet$ -- стягиваемый $\quad \Leftrightarrow \quad$ $K^\bullet$ -- расщипимый и ацикличный.
\end{to_suj}
\begin{to_suj}
$K^\bullet$ -- ограниченный расщипимый комплекс $\quad \Rightarrow \quad$ $K^\bullet\overset{qis}{\sim}H^\bullet(K^\bullet)$. \\
Иными словами любой расщипимый комплекс представляется в виде суммы комплексов вида \ref{eq:vid2} и \ref{eq:vid3}.
\end{to_suj}
\begin{to_def}
    Морфизм комплексов $f\colon K^\bullet \to L^\bullet$ называется гомотопической эквивалентностью, если $\exists f\colon L^\bullet \to K^\bullet$ такой что $fg \sim id_{K^\bullet}$ и $gf\sim id_{L^\bullet}$
\end{to_def}
\begin{to_suj}
\label{hom_eq_con}
 $f$ -- гомотопическая эквивалентность $\quad \Leftrightarrow \quad$ $C(f)$ -- стаягиваем.
\end{to_suj}\footnote{существует более прямое доказательство этого утверждения, получаемое перемножением соответсвующих матриц морфизмов, см. листочки}
    Воспользуемся введённым ранее понятием комплекса морфизмов \ref{hHom}.\footnote{Это конструкция представляет собой 'внутренний' $\hHom$. Обычный функтор $Hom_{\A}$ действует из $\A$ в кататегорию абелевых групп, а $\hHom$ из категории комплексов в категорию комплексов} 
    \[\hHom (K^\bullet, L^\bullet)^{i} = \prod_{n\in \Z}\Hom(K^n, L^{n+i})\]
    \[\partial\colon \quad g^m = df^m - (-1)^if^{m+1}d\]
    \[f\in \hHom(K^\bullet, L^\bullet)^i\]
    \[\partial\colon f \mapsto g\]
\begin{to_com}[Циклы комплекса морфизмов]\footnote{$i$-ые циклы комплекса морфизмов это морфизмы комплексов первого комплекса в сдвиг на $i$ второго}
    В случае $i=0$ циклами морфизма комплексов будут просто морфизмы комплексов 
    \[
        Z^0 (\hHom(K^\bullet, L^\bullet)) = \Hom(K^\bullet, L^\bullet)
    \]
    \[
    df^n = f^{n+1}d
    \]
    А для $i>0$ это будут морфизмы в сдвинутые комплексы, то есть 
    \[
    Z^i (\hHom(K^\bullet, L^\bullet)) = \Hom(K^\bullet, L^\bullet[i])
    \]
\end{to_com}
\begin{to_com}[Границы комплекса морфизмов]
    В случае $i=0$ выражение для дифференциала принимает вид $\partial\colon f^m\mapsto g^m = df^m - f^{m+1}d$. Образом такого отображения будет гомотопный нулю морфизм. При $i>0$ получим гомотопные нулю морфизмы в сдвинутый комплекс.
    \[
        B^i (\hHom(K^\bullet, L^\bullet)) = \lbrace f\in Hom(K^\bullet, L^\bullet[i])\quad|\quad f\sim 0 \rbrace
    \]
\end{to_com}
\begin{to_com}[Когомологии комплекса морфизмов]\label{cohomology_comp_mor}
Наконец, когомологиями комплекса морфизмов будут гомотопический классы эквивалентности морфизмов комплексов.
\[
H^i (\hHom(K^\bullet, L^\bullet)) = \Hom_{\K{A}}(K^\bullet, L^\bullet[i])
\]
\end{to_com}
Теперь можем доказывать утверждение \ref{hom_eq_con}.
\begin{proof}
Применим функтор $\hHom(X,)$ к короткой точной последовательности
\bee\label{3:cone_short_exact_eq}
    \begin{tikzcd}[sep=small]
	0 & L & {C(f)} & {K[1]} & 0
	\arrow[from=1-1, to=1-2]
	\arrow[from=1-2, to=1-3]
	\arrow[from=1-3, to=1-4]
	\arrow[from=1-4, to=1-5]
\end{tikzcd}
\eee
\bee
\begin{tikzcd}[sep=small]
	0 & \hHom(X, L) & \hHom(X, {C(f)}) & \hHom({K[1]}) & 0
	\arrow[from=1-1, to=1-2]
	\arrow[from=1-2, to=1-3]
	\arrow[from=1-3, to=1-4]
	\arrow[from=1-4, to=1-5]
\end{tikzcd}
\eee
Имея теперь короткую точную последовательность комплексов, естественно применить лемму о зиг-заге \ref{B:zigzag_lemma}. Получим длинную точную последовательность когомологий комплексов морфизмов, которые в силу \ref{cohomology_comp_mor} являются просто гомотопическими классами эквивалентности морфизмов в сдвинутый комплекс, то есть
\bee
\begin{tikzcd}[sep=small]
	\underset{0}{\Hom_{\K{A}}(X, {C(f)}[-1])} & \Hom_{\K{A}}(X, K) & \Hom_{\K{A}}(X, L) & \underset{0}{\Hom_{\K{A}}(X, {C(f)})} & \ldots
	\arrow[from=1-1, to=1-2]
	\arrow["\cong", from=1-2, to=1-3]
	\arrow[from=1-3, to=1-4]
	\arrow[from=1-4, to=1-5]
\end{tikzcd}
\eee
\end{proof}
\begin{to_lem}
    Короткой точной последовательности комплексов соответсвует длинная точная последовательность морфизмов в гомотопической категории
    \bee
\begin{tikzcd}[sep=small]
    0 \arrow[r]& K^\bullet \arrow[r]& L^\bullet \arrow[r]&M^\bullet \arrow[r]&0
\end{tikzcd}
    \eee
\bee
\begin{tikzcd}[sep=small]
     \ldots\arrow[r]& \Hom_{\K{A}}(X, K^\bullet) \arrow[r]& Hom_{\K{A}}(X,L^\bullet) \arrow[r]&\Hom_{\K{A}}(X, M^\bullet) \arrow[r]&\Hom_{\K{A}}(X, M^\bullet[1])\ldots
\end{tikzcd}
    \eee
\end{to_lem}
\begin{to_suj}[Тактическая цель]\label{tactics}
    Проективная резольвента -- строгий функтор
    \begin{align*}
        \mathcal{P}\colon&\A \to\K{A}\\
         &A \mapsto P_\bullet(A)
    \end{align*}
\end{to_suj}
\begin{to_lem}
    Пусть $P_\bullet$ -- проективная резольвента $M$, $K_\bullet$ -- какая-то резольвента $N$, также пусть есть морфизм $f \colon M \to N$. Тогда $\exists g_\bullet\colon P_\bullet\to K_\bullet$, такой, что $H^0(g_\bullet) = f$
\end{to_lem}
\begin{proof}
    Просто построим этот морфизм комплексов. Дополним проективный комплекс до ацикличного. Далее доказательство будем проводить по индукции.  В силу проективности можем поднять $f\varepsilon$ вдоль эпиморфизма из крайнего члена резольвенты $K_\bullet$ до морфизма $f_0$. Аналогично далее все морфизмы диагональные морфизмы вида $f_i\circ d_{i-1}^P$ поднимаются вдоль эпиморфизмов вида $d_{i-1}^K\colon K_{i-1}\twoheadrightarrow \im d_{i-1}^K$.
    \bee
        \begin{tikzcd}
            \ldots\arrow[r]& P_2\arrow[d, "f_2"] \arrow[r]&P_1 \arrow[d, "f_1"]\arrow[rd]\arrow[r, "d^P_1"]&P_0\arrow[rd, "f\varepsilon"] \arrow[r, "\varepsilon"]\arrow[d, dashed, "f_0"]&M\arrow[r]\arrow[d, "f"]&0\\
            \ldots\arrow[r]& K_2 \arrow[r]&K_1 \arrow[r, "d^K_1"']&K_0 \arrow[r, two heads]&N\arrow[r]&0
        \end{tikzcd}
    \eee
\end{proof} 
\begin{to_lem}
    Лбой морфизм из ограниченного справа комплекса из проективных объектов в ацикличный гомотопен нулю. 
\end{to_lem}
\begin{proof}
Крайний морфизм строится по проективности $P_0$ поднятием вдоль эпиморфизма на $\im d_1 = \ker d_{1}$. Получили базу индукции $f_0 = d_1 h_0 - 0 $.
        \bee
        \begin{tikzcd}
            \ldots\arrow[r]& P_2\arrow[d, "f_2"] \arrow[r]&P_1 \arrow[d, "f_1"]\arrow[r, "d^P_1"]&P_0\arrow[d, "f_0"]\arrow[ld, "h_0"]\arrow[r]&0\arrow[r]\arrow[d, "f_{-1}"]&\ldots\\
            \ldots\arrow[r]& K_2 \arrow[r]&K_1 \arrow[r, "d^K_1"']&K_0 \arrow[r]&K_{-1}\arrow[r]&\ldots
        \end{tikzcd}
    \eee
Пусть теперь все гомотопии до $i$-й построены и $f_{i-1} = h_{i-2}d - dh_{i-1}$ Рассмотри разность образов вертикальной и диагональной стрелок 
\[g_i = f_i - h_{i-1} d_{i+1} \]
\begin{align*}
       d_i g_i = d(f_i - h_{i-1}d) = f_{i-1}d - d h_{i-1}d = (h_{i-2}d - d h_{i-1})d - d h_{i-1}d = d h_{i-1} d - d h_{i-1} d = 0
\end{align*}
\end{proof}
\end{document}