\documentclass[../main.tex]{subfiles}
\begin{document}
\section{Локализация гомотопической категории и производная категория}
Вспомним определение \ref{DerCat} производной категории. 
Возникает вопрос о том, являются ли квазиизоморфизмы локализующим семейством. 
Не очень сложно проверить, что это не так.
\begin{to_suj}\label{Kom_Qis_counterex}
$Qis$ не явлется в общем случае локализующим семейством в категории $Kom(\mathcal{A})$.
\end{to_suj}
\begin{proof}
Возьмём $X\in Ob\mathcal{A}$ с инъективной резольвентой длины 1. 
Тогда естественно возникнет два вложенных в инъективную резольвенту комплекса, соответствующих объектам $X$ и $I_1$.
\begin{equation*}
\xymatrix{
\ldots\ar[r]&0\ar[r]\ar[d]&X\ar[r]\ar[d]_i&0\ar[d]\ar[r]&\ldots\\
0\ar[r]&X\ar[r]^i&I_0\ar[r]&I_1\ar[r]&0\\
&\ldots\ar[r]&0\ar[r]\ar[u]&I_1\ar[u]_{id}\ar[r]&0\ar[u]
}
\end{equation*}
Вложение комплекса $X[0]$ будет квазиизоморфизмом. 
Тогда второе условие Оре \eqref{ore_2_cond} гарантирует существование комплекса с квазиизоморфизмом в $I_1[-1]$ и морфизмом в $X[0]$. 
Естественно, из вида наших комплексов найденный объект имеет вид \xymatrix{0\ar[r]&P_0\ar[r]&P_1\ar[r]&0}. Получим коммутативную диаграмму.
\begin{equation*}
    \xymatrix{0\ar[r]&I_0\ar[r]&I_1\ar[r]&0\\
    0\ar[u]\ar[r]&0\ar[u]\ar[r]&I_1\ar[r]\ar[u]&0\ar[u]\\
    0\ar[u]\ar[r]&P_0\ar[r]\ar[u]&P_1\ar[r]\ar[u]&0\ar[u]}\quad
    \xymatrix{0\ar[r]&I_0\ar[r]&I_1\ar[r]&0\\
    0\ar[u]\ar[r]&X\ar[u]\ar[r]&0\ar[r]\ar[u]&0\ar[u]\\
    0\ar[u]\ar[r]&P_0\ar[r]\ar[u]&P_1\ar[r]\ar[u]&0\ar[u]}
\end{equation*}
Так как $P_\bullet\to I_1[-1]$ --- квазиизоморфизм, то $P_\bullet$ имеет лишь одну нетривиальную когомологию в первом члене. Отсюда $P_0\to I_0$ --- нулевой морфизм.
С другой стороны морфизм $P_0\to X$ в общем случае ненулевой. Также ненулевым является $X\to I_0$ из квазиизоморфности. Противоречие заключается в коммутативности следующей диаграммы.
\begin{equation*}
    \xymatrix{&0\ar[rd]\\
    P_0\ar[ru]\ar[rd]&&I_0\\
    &X\ar[ru]}
\end{equation*}
\end{proof}
Несмотря на утверждение выше, описать структуру производной категории $\mathcal{D}(\mathcal{A})$ можно. Для начала нашей целью будет установить следующую теорему.
{\color{red} Это надо переписать после того как напишу про конусы в начале}
\begin{to_thr}
$Qis$ --- локализующее семейство в $\mathcal{K}(\mathcal{A})$.
\end{to_thr}
\begin{proof}
Первое условие Оре очевидно. Проверим второе \eqref{ore_2_cond}.
Рассмотрим тройку $X, Y, Z\in Ob\mathcal{K}(\mathcal{A})$ с морфизмами $f\in \Hom_\mathcal{A}(X, Z)$, $s\in \Hom_\mathcal{A}(Y, Z)$, где $s\in Qis$ из локализующего семейства. Объектом, который будет удовлетворять условию, станет сдвинутый конус $C[hf][-1]$, где $h$ --- естественный морфизм из $Z$ в конус $C[s]$.
    \begin{equation}\label{qis_loc_diagram}
        \begin{tikzcd}[sep=small]
	{C[hf][-1]} & Y \\
	X & {Z} \\
	& {C[s]}
	\arrow["s", from=1-2, to=2-2]
	\arrow["f", from=2-1, to=2-2]
	\arrow["h", from=2-2, to=3-2]
	\arrow["hf"{description}, from=2-1, to=3-2]
	\arrow["\alpha", from=1-1, to=2-1]
	\arrow["\beta", from=1-1, to=1-2]
\end{tikzcd}
\end{equation}
Рассмотрим откуда берутся морфизмы $\alpha, \beta$ в \eqref{qis_loc_diagram} и почему $\alpha \in Qis$.
Напомним, что вместе с конусом $C[s]$ поставляется короткая точная последовательность комплексов:
\begin{equation*}
        \begin{tikzcd}[sep=small]
	0&C[s] & {C[hf]} & {X[1]}&0
	\arrow[from=1-2, to=1-3]
	\arrow[from=1-3, to=1-4]
 \arrow[from=1-1, to=1-2]
 \arrow[from=1-4, to=1-5]
\end{tikzcd}
    \end{equation*}
    которая вместе со сдвигом на 1 дает морфизм $\alpha$ в \eqref{qis_loc_diagram}. Применим к такой сдвинутой последовательности лемму о зигзаге. Так как $s\in Qis$, у $C[s]$ нулевые когомологии. Отсюда получим точные последовательности.
    \begin{equation}\label{32_snake_lemma}
        \begin{tikzcd}
	0 & {H_i(C[hf][-1])} & {H_i(X)} & 0
	\arrow[from=1-1, to=1-2]
	\arrow["{\alpha^*}", from=1-2, to=1-3]
	\arrow[from=1-3, to=1-4]
\end{tikzcd}
    \end{equation}
Тогда $\alpha \in Qis$. Для поиска $\beta$ такого, что $s\beta$ гомотопен $f\alpha$, напомним также о длинной точной последовательности, поставляющейся с конусом.
    \begin{equation}\label{32_long_cone}
        \begin{tikzcd}[sep=small]
	\ldots&Y&Z& {C[s]} & Y[1]&Z[1]&\ldots
	\arrow[from=1-2, to=1-3]
	\arrow[from=1-3, to=1-4]
 \arrow[from=1-1, to=1-2]
 \arrow[from=1-4, to=1-5]
 \arrow[from=1-5, to=1-6]
 \arrow[from=1-6, to=1-7]
\end{tikzcd}
    \end{equation}
Воспользуемся следующим утверждением для такой последовательности, которое докажем немного позже.

\begin{to_lem}\label{triangulation_33}
Композиция любых двух морфизмов в последовательности \eqref{32_long_cone} гомотопна нулю.
\end{to_lem}
Для такой последовательности нас интересует т. н. \emph{выделенный треугольник}: точная подпоследовательность \eqref{32_long_cone} из четырёх элементов. Применяя функтор $\Hom_{\mathcal{K}(\mathcal{A})}(C[hf][-1], \cdot)$ получим также точную последовательность множеств морфизмов из конуса.
    \begin{equation*}
       \begin{tikzcd}[sep=small]
	\ldots & Y & Z & {C[s]} & \ldots \\
	\ldots & {\Hom_{\mathcal{K}(\mathcal{A})}(C[hf][-1], Y)} & {\Hom_{\mathcal{K}(\mathcal{A})}(C[hf][-1], Z)} & {\Hom_{\mathcal{K}(\mathcal{A})}(C[hf][-1], C[s])} & \ldots \\
	& \beta & f\alpha & hf\alpha
	\arrow[from=1-2, to=1-3]
	\arrow[from=1-3, to=1-4]
	\arrow["{\Hom_{\mathcal{K}(\mathcal{A})}(C[hf][-1], \cdot)}", maps to, from=1-3, to=2-3]
	\arrow[from=2-2, to=2-3]
	\arrow[from=2-3, to=2-4]
	\arrow[maps to, from=3-3, to=3-4]
	\arrow[maps to, from=3-2, to=3-3]
	\arrow[from=1-4, to=1-5]
	\arrow[from=1-1, to=1-2]
	\arrow[from=2-1, to=2-2]
	\arrow[from=2-4, to=2-5]
\end{tikzcd}
    \end{equation*}
Применяя утверждение \ref{triangulation_33} к аналогичной \eqref{32_long_cone} последовательности для конуса $C[hf]$ получим, что $hf\alpha \sim 0$. Отсюда из точности этот порфизм поднимается до $\beta \in \Hom_{\mathcal{K}(\mathcal{A})}(C[hf][-1], Y)$.
Докажем выполнение третьего условия Оре. Нам пригодится еще одно утверждение.
\begin{to_lem}\label{32_additive_loc}
Локализация аддитивной категории локализующим семейством тоже аддитивна.
\end{to_lem}
В силу аддитивности небоходимо доказать, что $sf\sim 0$, где $s$ из локализующего семейства, влечёт существование $t\in Qis: ft\sim 0$.
Конструкция будет следующей. Опять возьмём конус $C[s]$, соответствующую точную последовательность и применим функтор $\Hom_{\mathcal{K}(\mathcal{A})}(X, \cdot)$.
    \begin{equation*}
        \begin{tikzcd}[sep=small]
	&& X&C[f'][-1] \\
	\ldots & {C[s][-1]} & Y & Z & \ldots \\
	\ldots & {\Hom_{\mathcal{K}(\mathcal{A})}(X, C[s][-1])} & {\Hom_{\mathcal{K}(\mathcal{A})}(X, Y)} & {\Hom_{\mathcal{K}(\mathcal{A})}(X, Z)} & \ldots \\
	& {f'} & f & {sf\sim 0}
    \arrow[from=1-4, to=1-3, "t"]
	\arrow["h"{description}, from=2-2, to=2-3]
	\arrow["s", from=2-3, to=2-4]
	\arrow[from=2-4, to=2-5]
	\arrow[from=2-1, to=2-2]
	\arrow["{\Hom_{\mathcal{K}(\mathcal{A})}(X, \cdot)}", maps to, from=2-3, to=3-3]
	\arrow[from=3-2, to=3-3]
	\arrow[from=3-1, to=3-2]
	\arrow[from=3-3, to=3-4]
	\arrow[from=3-4, to=3-5]
	\arrow["f"', from=1-3, to=2-3]
	\arrow[maps to, from=4-3, to=4-4]
	\arrow[maps to, from=4-2, to=4-3]
    \arrow[from=1-3, to=2-2, "f'"{description}]
\end{tikzcd}
    \end{equation*}
Из точности $sf$ поднимается до $f'\in \Hom_{\mathcal{K}(\mathcal{A})}(X, C[s][-1])$. Искомым морфизмом тогда будет соответствующий канонический для конуса $t:C[f'][-1]\to X$.
Для такого построения остается проверить квазиизоморфность и гомотопность композиции нулю. Аналогично квазиизоморфность следует из ацикличности конуса $C[s]$ и точности следующей последовательности по лемме о зигзаге.
\begin{equation*}
    \begin{tikzcd}[sep=small]
	\ldots & {C[f'][-1]} & X & {C[s][-1]} & {C[f']} & \ldots
	\arrow["t", from=1-2, to=1-3]
	\arrow["f'", from=1-3, to=1-4]
	\arrow[from=1-4, to=1-5]
	\arrow[from=1-5, to=1-6]
	\arrow[from=1-1, to=1-2]
\end{tikzcd}
\end{equation*}
Гомотопность нулю следует из того, что по построению $ft\sim hf't$, а $f't\sim 0$ из точности последовательности выше.
\end{proof}
\begin{proof}[Доказательство леммы \ref{triangulation_33}]
    i dont know
\end{proof}
\begin{proof}[Доказательство леммы \ref{32_additive_loc}]
    i dont know
\end{proof}
Ещё раз вспомним логику всего, что делалось ранее. Нашей целью было отождествить объект со всеми его резольвентами. Оказалось, что все резольвенты квазиизоморфны объекту: поэтому мы захотели изучить локализацию категории комплексов квазиизоморфозмами, ведь в такой категории интересующее нас отношение является изоморфизмом. Но такая локализация простому описанию сразу не поддаётся, в чём мы убедились в предложении \ref{Kom_Qis_counterex}. Поэтому сначала мы научились обращать квазиизоморфизмы в гомотопической категории. Следующая теорема завершает наши рассуждения, утверждая, что получившаяся локализация в гомотопической категории и производная категория --- это одно и то же.
\begin{to_thr}
\begin{equation*}
    \mathcal{D}(\mathcal{A}) \cong \mathcal{K}(\mathcal{A})[Qis^{-1}]
\end{equation*}
\end{to_thr}
\begin{proof}
    \begin{equation*}
        \begin{tikzcd}
	{Kom(\mathcal{A})} & {\mathcal{D}(\mathcal{A})} \\
	{\mathcal{K}(\mathcal{A})} & {\mathcal{K}(\mathcal{A})[Qis^{-1}]}
	\arrow["{Q_\mathcal{K}}", from=1-1, to=2-1]
	\arrow["{Q_\mathcal{D}}", from=1-1, to=1-2]
	\arrow["{Q_\mathcal{D}}"', from=2-1, to=2-2]
	\arrow["\beta"', bend right, dashed, from=1-2, to=2-2]
	\arrow["\alpha"', bend right, dashed, from=2-2, to=1-2]
 \arrow["\pi"{description}, dashed, from=2-1, to=1-2]
\end{tikzcd}
    \end{equation*}
Покажем, что существуют единственные два морфизма $\alpha, \beta$, делающие диаграмму выше коммутативной. 
Функтор $Q_\mathcal{K}$ обращает гомотопические эквивалентности (отметим, что гомотопическая категория --- локализация категории комплексов гомотопическими эквивалентностями) и сохраняет квазиизоморфизмы. $Q_\mathcal{D}$ обращает все квазиизоморфизмы. Тогда и $Q_\mathcal{R}\cdot Q_\mathcal{D}$ обращает все квазиизоморфизмы. Следовательно по универсальному свойству локализации \eqref{loc_univ} существует единственный $\beta$.
Отметим, что все гомотопические эквивалентности --- это квазиизоморфизмы. Следовательно, $Q_\mathcal{D}$ переводит все гомотопические эквивалентности в изоморфизмы и опять по универсальному свойству локализации \eqref{loc_univ} имеем единственный морфизм $\pi$. Из коммутативности $\pi$ переводит все квазиизоморфизмы в изоморфизмы. Ещё раз по универсальному свойству \eqref{loc_univ} имеем единственный морфизм $\alpha$.
\end{proof}
\end{document}