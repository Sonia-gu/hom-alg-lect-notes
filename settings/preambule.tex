% file's preambule
% connect packages
%%%%%%%%%%%%%%%%%%%%%%%%%%%%%%%%%%%%%%%%%%%%%%%%%%%%%%%%%%%%%%%%%%%%%%

\usepackage[T2A]{fontenc}                   %!? закрепляет внутреннюю кодировку LaTeX
\usepackage[utf8]{inputenc}                 %!  закрепляет кодировку utf8
\usepackage[english,russian]{babel}         %!  подключает русский и английский
\usepackage{amsmath}
\usepackage{amssymb}
\usepackage{amssymb,textcomp, esvect,esint} %!  |важно для формул 
\usepackage[margin=1.5cm]{geometry}           %!  фиксирует оступ на 2cm
\usepackage{amsfonts}                       %!  математические шрифты
\usepackage{amsthm}                         %!  newtheorem и их сквозная нумерация
\usepackage{graphicx}                       %?  графическое изменение текста
\usepackage{indentfirst}                    %   добавить indent перед первым параграфом
\usepackage{xcolor}                         %   добавляет цвета
\usepackage{enumitem}                       %!  задание макета перечня.
\usepackage[unicode, pdftex]{hyperref}      %!  оглавление для панели навигации по PDF-документу + гиперссылки
\usepackage{booktabs}                       %!  добавляет книжные линии в таблицы
\usepackage{hypcap}                         %?  адресация на картинку, а не на подпись к ней
\usepackage{abraces}                        %?  фигурные скобки сверху или снизу текста
\usepackage{caption}                        %-  позволяет корректировать caption 
\usepackage{multirow}                       %   объединение ячеек в таблицах
\usepackage{pifont}                         %!  нужен для крестика
\usepackage{cancel}                         %!  аутентичное перечеркивание текста
\usepackage{ulem}                           %!  перечеркивание текста
\usepackage{tikz}                           %!  высокоуровневые рисунки (кружочек)
\usepackage{titling}                        %-  автоматическое заглавие 
\usepackage{blindtext}                      %-  слепой текст
\usepackage{fancyhdr}                       %   добавить верхний и нижний колонтитул

\usepackage{fancyhdr}
\usepackage{accents}
\pagestyle{fancy}
\setlength{\headheight}{40pt}

\usepackage{import}
\usepackage{xifthen}
\usepackage{pdfpages}
\usepackage{transparent}

\usepackage{mathrsfs}  

\usepackage[yyyymmdd]{datetime} %yyy/mm/dd
\renewcommand{\dateseparator}{/}

%epigraph
\usepackage{tcolorbox}
%%%%%%%%%%%%%%%%%%%%%%%%%%%%%%%%%%%%%%%%%%%%%%%%%%%%%%%%%%%%%%%%%%%%%%
%subfiles


\usepackage{subfiles} % Best loaded last in the preambles

%%%%%%%%%%%%%%%%%%%%%%%%%%%%%%%%%%%%%%%%%%%%%%%%%%%%%%%%%%%%%%%%%%%%%%
\usepackage{tikz-cd}
\usetikzlibrary{cd, babel}
\usepackage[all, cmtip]{xy} %ВПИЗДУ ЭТОТ ТАКЗ БЛЯТЬ // САМ ИДИ В ПИЗДУ БЛЯ(миша)// ЭЭЭЭ ТЫ АХУЕЛ БЛЯ???
\usetikzlibrary{cd}

\usepackage{eulervm}
\DeclareMathOperator{\Hom}{Hom}% preferred

% create environments
%%%%%%%%%%%%%%%%%%%%%%%%%%%%%%%%%%%%%%%%%%%%%%%%%%%%%%%%%%%%%%%%%%%%%%

\newtheorem{to_thr}{Thr}[section]
\newtheorem{to_suj}[to_thr]{Prop}
\newtheorem{to_lem}[to_thr]{Lem}
\newtheorem{to_law}[to_thr]{Law}
\newtheorem{to_com}[to_thr]{Com}
\newtheorem{to_con}[to_thr]{Corr}
\newtheorem{to_ex}[to_thr]{Ex}
\newtheorem{to_claim}[to_thr]{Claim}
\newtheorem{to_note}[to_thr]{Note}
\theoremstyle{definition}
\newtheorem{to_def}[to_thr]{def}
\renewcommand\qedsymbol{$\blacksquare$}


%%%%%%%%%%%%%%%%%%%%%%%%%%%%%%%%%%%%%%%%%%%%%%%%%%%%%%%%%%%%%%%%%%%%%%

\newcommand{\incfig}[1]{%
    % \def\svgwidth{\columnwidth}
    \import{./figures/}{#1.pdf_tex}
}
%%%%%%%%%%%%%%%%%%%%%%%%%%%%%%%%%%%%%%%%%%%%%%%%%%%%%%%%%%%%%%%%%%%%%%
%add
\newcommand*{\be}{\begin{equation}}
\newcommand*{\ee}{\end{equation}}
\newcommand*{\bee}{\begin{equation*}}
\newcommand*{\eee}{\end{equation*}}
\newcommand*{\wt}{\widetilde}
\newcommand{\R}{\mathbb{R}}
\newcommand{\N}{\mathbb{N}}
\newcommand{\Z}{\mathbb{Z}}
\renewcommand{\C}{\mathbb{C}}
\newcommand{\Q}{\mathbb{Q}}

\renewcommand{\Im}{\mathop{\mathrm{Im}}\nolimits}
\renewcommand{\Re}{\mathop{\mathrm{Re}}\nolimits}
\renewcommand{\div}{\mathop{\mathrm{div}}\nolimits}
\renewcommand{\d}{\, d}
\renewcommand{\leq}{\leqslant}
\renewcommand{\geq}{\geqslant}

\newcommand{\pd}{\mathbf{pd}\text{ }}
\newcommand{\gldim}{\mathbf{gldim}\text{ }}
\newcommand{\A}{\mathcal{A}}
\newcommand{\B}{\mathcal{B}}
\newcommand{\F}{\mathcal{F}}
\newcommand{\Ab}{\mathbf{Ab}}
\newcommand{\K}[1]{\mathcal{K(#1)}}
\newcommand{\Kstar}[1]{\mathcal{K^{*}(#1)}}
\newcommand{\Kl}[1]{\mathcal{K^+(#1)}}
\newcommand{\KlF}[1]{\mathcal{K^+ #1}}
\newcommand{\Kr}[1]{\mathcal{K^-(#1)}}
\newcommand{\Kb}[1]{\mathcal{K^\flat(#1)}}
\newcommand{\Dl}[1]{\mathcal{D^+(#1)}}
\newcommand{\DlF}[1]{\mathcal{D^+ #1}}
\newcommand{\DrF}[1]{\mathcal{D^- #1}}
\newcommand{\DF}[1]{\mathcal{D #1}}
\newcommand{\Dr}[1]{\mathcal{D^-(#1)}}
\newcommand{\D}[1]{\mathcal{D(#1)}}
\newcommand{\Dstar}[1]{\mathcal{D^{*}(#1)}}
\newcommand{\Db}[1]{\mathcal{D^\flat (#1)}}
\newcommand{\Rn}[2]{\mathcal{R}^{#1}\mathcal{#2}}
\newcommand{\Ln}[2]{\mathcal{L}^{#1}\mathcal{#2}}
\newcommand{\hHom}{\underline{Hom}}
\newcommand{\taule}[1]{\tau_{ \curlyeqprec #1}}
\newcommand{\tauge}[1]{\tau_{ \curlyeqsucc #1}}
\newcommand\NB[1][0.3]{N\kern-#1em B}

\newcommand{\vc}[1]{\mbox{\boldmath $#1$}}
\newcommand{\T}{^{\text{T}}}

\newcommand{\grad}{\mathop{\mathrm{grad}}\nolimits}
\newcommand{\rot}{\mathop{\mathrm{rot}}\nolimits}
\newcommand{\diag}{\mathop{\mathrm{diag}}\nolimits}
\newcommand{\Ker}{\mathop{\mathrm{Ker}}\nolimits}
\newcommand{\Spec}{\mathop{\mathrm{Spec}}\nolimits}
\newcommand{\sign}{\mathop{\mathrm{sign}}\nolimits}
\newcommand{\tr}{\mathop{\mathrm{tr}}\nolimits}
\newcommand{\rg}{\mathop{\mathrm{rg}}\nolimits}

\newcommand{\const}{\text{const}}
\newcommand{\red}[1]{\textcolor{red}{#1}}
\newcommand{\xmark}{\ding{55}}

\renewcommand{\int}{\upint}
\renewcommand{\oint}{\upoint}

\makeatletter
\newcommand{\colim@}[2]{%
  \vtop{\m@th\ialign{##\cr
    \hfil$#1\operator@font colim$\hfil\cr
    \noalign{\nointerlineskip\kern1.5\ex@}#2\cr
    \noalign{\nointerlineskip\kern-\ex@}\cr}}%
}
\newcommand{\colim}{%
  \mathop{\mathpalette\colim@{\rightarrowfill@\scriptscriptstyle}}\nmlimits@
}
\renewcommand{\varprojlim}{%
  \mathop{\mathpalette\varlim@{\leftarrowfill@\scriptscriptstyle}}\nmlimits@
}
\renewcommand{\varinjlim}{%
  \mathop{\mathpalette\varlim@{\rightarrowfill@\scriptscriptstyle}}\nmlimits@
}

\DeclareMathOperator{\coker}{Coker}

\makeatother
%%%%%%%%%%%%%%%%%%%%%%%%%%%%%%%%%%%%%%%%%%%%%%%%%%%%%%%%%%%%%%%%%%%%%%
%links
\usepackage{xcolor}
\usepackage{hyperref}
\usepackage{floatrow}
\usepackage{wrapfig}

\definecolor{urlcolor}{HTML}{799B03} % hyperlink color
\hypersetup{pdfstartview=FitH,  linkcolor=blue,urlcolor=blue, colorlinks=true}

%Форматирование секций.
\usepackage{titlesec}

\titleformat{\section}[runin]{\bfseries}{\begin{tcolorbox}\small{Семинар }\thesection\\ }{0.5pt}{(Темы: }[)\end{tcolorbox}]

\titleformat{name=\section, numberless} %Форматирование заголовка оглавления
 {\bfseries}
 {}
 {0pt}
 {}
 
%%%%%%%%%%%%%%%%%%%%%%%%%%%%%%%%%%%%%%%%%%%%%%%%%%%%%%%%%%%%%%%%%%%%%